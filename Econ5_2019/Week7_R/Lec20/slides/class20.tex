\documentclass[ignorenonframetext,]{beamer}
\setbeamertemplate{caption}[numbered]
\setbeamertemplate{caption label separator}{: }
\setbeamercolor{caption name}{fg=normal text.fg}
\beamertemplatenavigationsymbolsempty
\usepackage{lmodern}
\usepackage{amssymb,amsmath}
\usepackage{ifxetex,ifluatex}
\usepackage{fixltx2e} % provides \textsubscript

\usetheme{CambridgeUS}
\definecolor{UBCblue}{rgb}{0.04706, 0.13725, 0.26667}
\definecolor{UBCgray}{rgb}{0.3686, 0.5255, 0.6235}
\colorlet{verylightgray}{gray!10}
\setbeamercolor{palette primary}{bg=UBCblue,fg=white}
\setbeamercolor{palette secondary}{bg=darkgray,fg=white}
\setbeamercolor{palette tertiary}{bg=UBCblue,fg=white}
\setbeamercolor{palette quaternary}{bg=UBCblue,fg=white}
\setbeamercolor{structure}{fg=UBCblue} % itemize, enumerate, etc
\setbeamercolor{section in toc}{fg=UBCblue} % TOC sections
\setbeamercolor{subsection in head/foot}{bg=darkgray,fg=white}
\setbeamercolor{frametitle}{fg=UBCblue}
\setbeamercolor{title}{fg=UBCblue, bg=verylightgray}
\setbeamertemplate{itemize items}{\color{UBCblue}$\blacktriangleright$}


\ifnum 0\ifxetex 1\fi\ifluatex 1\fi=0 % if pdftex
\usepackage[T1]{fontenc}
\usepackage[utf8]{inputenc}
\else % if luatex or xelatex
\ifxetex
\usepackage{mathspec}
\else
\usepackage{fontspec}
\fi
\defaultfontfeatures{Ligatures=TeX,Scale=MatchLowercase}
\fi
% use upquote if available, for straight quotes in verbatim environments
\IfFileExists{upquote.sty}{\usepackage{upquote}}{}
% use microtype if available
\IfFileExists{microtype.sty}{%
	\usepackage{microtype}
	\UseMicrotypeSet[protrusion]{basicmath} % disable protrusion for tt fonts
}{}
\newif\ifbibliography
\hypersetup{
	pdfauthor={UCSD},
	pdfborder={0 0 0},
	breaklinks=true}
\urlstyle{same}  % don't use monospace font for urls
\usepackage{color}
\usepackage{fancyvrb}
\newcommand{\VerbBar}{|}
\newcommand{\VERB}{\Verb[commandchars=\\\{\}]}
\DefineVerbatimEnvironment{Highlighting}{Verbatim}{commandchars=\\\{\}}
% Add ',fontsize=\small' for more characters per line
\usepackage{framed}
\definecolor{shadecolor}{RGB}{248,248,248}
\newenvironment{Shaded}{\begin{snugshade}}{\end{snugshade}}
\newcommand{\KeywordTok}[1]{\textcolor[rgb]{0.13,0.29,0.53}{\textbf{#1}}}
\newcommand{\DataTypeTok}[1]{\textcolor[rgb]{0.13,0.29,0.53}{#1}}
\newcommand{\DecValTok}[1]{\textcolor[rgb]{0.00,0.00,0.81}{#1}}
\newcommand{\BaseNTok}[1]{\textcolor[rgb]{0.00,0.00,0.81}{#1}}
\newcommand{\FloatTok}[1]{\textcolor[rgb]{0.00,0.00,0.81}{#1}}
\newcommand{\ConstantTok}[1]{\textcolor[rgb]{0.00,0.00,0.00}{#1}}
\newcommand{\CharTok}[1]{\textcolor[rgb]{0.31,0.60,0.02}{#1}}
\newcommand{\SpecialCharTok}[1]{\textcolor[rgb]{0.00,0.00,0.00}{#1}}
\newcommand{\StringTok}[1]{\textcolor[rgb]{0.31,0.60,0.02}{#1}}
\newcommand{\VerbatimStringTok}[1]{\textcolor[rgb]{0.31,0.60,0.02}{#1}}
\newcommand{\SpecialStringTok}[1]{\textcolor[rgb]{0.31,0.60,0.02}{#1}}
\newcommand{\ImportTok}[1]{#1}
\newcommand{\CommentTok}[1]{\textcolor[rgb]{0.56,0.35,0.01}{\textit{#1}}}
\newcommand{\DocumentationTok}[1]{\textcolor[rgb]{0.56,0.35,0.01}{\textbf{\textit{#1}}}}
\newcommand{\AnnotationTok}[1]{\textcolor[rgb]{0.56,0.35,0.01}{\textbf{\textit{#1}}}}
\newcommand{\CommentVarTok}[1]{\textcolor[rgb]{0.56,0.35,0.01}{\textbf{\textit{#1}}}}
\newcommand{\OtherTok}[1]{\textcolor[rgb]{0.56,0.35,0.01}{#1}}
\newcommand{\FunctionTok}[1]{\textcolor[rgb]{0.00,0.00,0.00}{#1}}
\newcommand{\VariableTok}[1]{\textcolor[rgb]{0.00,0.00,0.00}{#1}}
\newcommand{\ControlFlowTok}[1]{\textcolor[rgb]{0.13,0.29,0.53}{\textbf{#1}}}
\newcommand{\OperatorTok}[1]{\textcolor[rgb]{0.81,0.36,0.00}{\textbf{#1}}}
\newcommand{\BuiltInTok}[1]{#1}
\newcommand{\ExtensionTok}[1]{#1}
\newcommand{\PreprocessorTok}[1]{\textcolor[rgb]{0.56,0.35,0.01}{\textit{#1}}}
\newcommand{\AttributeTok}[1]{\textcolor[rgb]{0.77,0.63,0.00}{#1}}
\newcommand{\RegionMarkerTok}[1]{#1}
\newcommand{\InformationTok}[1]{\textcolor[rgb]{0.56,0.35,0.01}{\textbf{\textit{#1}}}}
\newcommand{\WarningTok}[1]{\textcolor[rgb]{0.56,0.35,0.01}{\textbf{\textit{#1}}}}
\newcommand{\AlertTok}[1]{\textcolor[rgb]{0.94,0.16,0.16}{#1}}
\newcommand{\ErrorTok}[1]{\textcolor[rgb]{0.64,0.00,0.00}{\textbf{#1}}}
\newcommand{\NormalTok}[1]{#1}
\usepackage{graphicx,grffile}
\makeatletter
\def\maxwidth{\ifdim\Gin@nat@width>\linewidth\linewidth\else\Gin@nat@width\fi}
\def\maxheight{\ifdim\Gin@nat@height>\textheight0.8\textheight\else\Gin@nat@height\fi}
\makeatother
% Scale images if necessary, so that they will not overflow the page
% margins by default, and it is still possible to overwrite the defaults
% using explicit options in \includegraphics[width, height, ...]{}
\setkeys{Gin}{width=\maxwidth,height=\maxheight,keepaspectratio}

% Prevent slide breaks in the middle of a paragraph:
\widowpenalties 1 10000
\raggedbottom

\AtBeginPart{
	\let\insertpartnumber\relax
	\let\partname\relax
	\frame{\partpage}
}
\AtBeginSection{
	\ifbibliography
	\else
	\let\insertsectionnumber\relax
	\let\sectionname\relax
	\frame{\sectionpage}
	\fi
}
\AtBeginSubsection{
	\let\insertsubsectionnumber\relax
	\let\subsectionname\relax
	\frame{\subsectionpage}
}

\setlength{\parindent}{0pt}
\setlength{\parskip}{6pt plus 2pt minus 1pt}
\setlength{\emergencystretch}{3em}  % prevent overfull lines
\providecommand{\tightlist}{%
	\setlength{\itemsep}{0pt}\setlength{\parskip}{0pt}}
\setcounter{secnumdepth}{0}


\title[Class 20]{Introduction to Social Data Analytics\\
	Class 20}
\author[Kaushik]{Arushi Kaushik}
\institute[UCSD]{arkaushi@ucsd.edu}
%\date[]{}



\begin{document}
\frame{\titlepage}

\begin{frame}[fragile]{Today: more practice with loops}

By the end of today's lecture, you should be able to:

\begin{itemize}
\tightlist
\item
  Build loops that utilize the `counter' (e.g.~i) in three ways:

  \begin{itemize}
  \tightlist
  \item
    As a number for calculations
  \item
    As a subset index
  \item
    As an element of a vector (of strings or numbers)
  \end{itemize}
\item
  Construct \texttt{while} loops that may include conditional statements
\end{itemize}

Open class20.R if you haven't already.

\end{frame}

\begin{frame}[fragile]{Tools in your loop toolbox}

\begin{itemize}
\tightlist
\item
  \texttt{for}, \texttt{if}, \texttt{else}, and (soon) \texttt{while}
\item
  Using \texttt{i} as a number itself
\item
  Using \texttt{i} to subset data
\item
  Using \texttt{i} as an element of a vector
\end{itemize}

\end{frame}

\begin{frame}[fragile]{The problem: no \texttt{year} vector}

From pre class exercise:

\begin{itemize}
\tightlist
\item
  Suppose I have data on the US population from 1790 - 1970
\item
  But I don't have a \texttt{year} vector
\end{itemize}

\begin{Shaded}
\begin{Highlighting}[]
\KeywordTok{data}\NormalTok{(uspop)}
\KeywordTok{summary}\NormalTok{(uspop)}
\end{Highlighting}
\end{Shaded}

\begin{verbatim}
##    Min. 1st Qu.  Median    Mean 3rd Qu.    Max. 
##    3.93   15.00   50.20   69.77  114.25  203.20
\end{verbatim}

\begin{Shaded}
\begin{Highlighting}[]
\NormalTok{n <-}\StringTok{ }\KeywordTok{length}\NormalTok{(uspop)}
\end{Highlighting}
\end{Shaded}

\begin{itemize}
\tightlist
\item
  How could I generate a \texttt{year} vector using loops?
\end{itemize}

\end{frame}

\begin{frame}[fragile]{Option 1: use \texttt{i} as a number itself}

\begin{Shaded}
\begin{Highlighting}[]
\ControlFlowTok{for}\NormalTok{(i }\ControlFlowTok{in} \DecValTok{1}\OperatorTok{:}\NormalTok{n)\{}
  \KeywordTok{print}\NormalTok{(}\DecValTok{1780} \OperatorTok{+}\StringTok{ }\DecValTok{10}\OperatorTok{*}\NormalTok{i)}
\NormalTok{\}}
\end{Highlighting}
\end{Shaded}

\end{frame}

\begin{frame}[fragile]{Option 1: use \texttt{i} as a number itself}

\begin{Shaded}
\begin{Highlighting}[]
\ControlFlowTok{for}\NormalTok{(i }\ControlFlowTok{in} \DecValTok{1}\OperatorTok{:}\NormalTok{n)\{}
  \KeywordTok{print}\NormalTok{(}\DecValTok{1780} \OperatorTok{+}\StringTok{ }\DecValTok{10}\OperatorTok{*}\NormalTok{i)}
\NormalTok{\}}
\end{Highlighting}
\end{Shaded}

\begin{verbatim}
## [1] 1790
## [1] 1800
## [1] 1810
## [1] 1820
## [1] 1830
## [1] 1840
## [1] 1850
## [1] 1860
## [1] 1870
## [1] 1880
## [1] 1890
## [1] 1900
## [1] 1910
## [1] 1920
## [1] 1930
## [1] 1940
## [1] 1950
## [1] 1960
## [1] 1970
\end{verbatim}

\end{frame}

\begin{frame}[fragile]{Option 2: use \texttt{i} to subset data}

\begin{Shaded}
\begin{Highlighting}[]
\NormalTok{year <-}\StringTok{ }\KeywordTok{rep}\NormalTok{(}\OtherTok{NA}\NormalTok{, n)}
\ControlFlowTok{for}\NormalTok{(i }\ControlFlowTok{in} \DecValTok{1}\OperatorTok{:}\NormalTok{n)\{}
\NormalTok{  year[i] <-}\StringTok{ }\DecValTok{1780} \OperatorTok{+}\StringTok{ }\DecValTok{10}\OperatorTok{*}\NormalTok{i}
  \KeywordTok{print}\NormalTok{(year[i])}
\NormalTok{\}}
\end{Highlighting}
\end{Shaded}

\end{frame}

\begin{frame}[fragile]{Option 2: use \texttt{i} to subset data}

\begin{Shaded}
\begin{Highlighting}[]
\NormalTok{year <-}\StringTok{ }\KeywordTok{rep}\NormalTok{(}\OtherTok{NA}\NormalTok{, n)}
\ControlFlowTok{for}\NormalTok{(i }\ControlFlowTok{in} \DecValTok{1}\OperatorTok{:}\NormalTok{n)\{}
\NormalTok{  year[i] <-}\StringTok{ }\DecValTok{1780} \OperatorTok{+}\StringTok{ }\DecValTok{10}\OperatorTok{*}\NormalTok{i}
  \KeywordTok{print}\NormalTok{(year[i])}
\NormalTok{\}}
\end{Highlighting}
\end{Shaded}

\begin{verbatim}
## [1] 1790
## [1] 1800
## [1] 1810
## [1] 1820
## [1] 1830
## [1] 1840
## [1] 1850
## [1] 1860
## [1] 1870
## [1] 1880
## [1] 1890
## [1] 1900
## [1] 1910
## [1] 1920
## [1] 1930
## [1] 1940
## [1] 1950
## [1] 1960
## [1] 1970
\end{verbatim}

\end{frame}

\begin{frame}[fragile]{Option 3: \texttt{i} as an element of a vector}

\begin{Shaded}
\begin{Highlighting}[]
\NormalTok{year <-}\StringTok{ }\KeywordTok{seq}\NormalTok{(}\DecValTok{1790}\NormalTok{, }\DecValTok{1970}\NormalTok{, }\DecValTok{10}\NormalTok{)}
\ControlFlowTok{for}\NormalTok{(i }\ControlFlowTok{in}\NormalTok{ year)\{}
  \KeywordTok{print}\NormalTok{(i)}
\NormalTok{\}}
\end{Highlighting}
\end{Shaded}

\begin{itemize}
\tightlist
\item
  This use is useful when \texttt{i} refers to an element of a vector of
  text strings
\end{itemize}

\end{frame}

\begin{frame}[fragile]{Option 3: \texttt{i} as an element of a vector}

\begin{Shaded}
\begin{Highlighting}[]
\NormalTok{year <-}\StringTok{ }\KeywordTok{seq}\NormalTok{(}\DecValTok{1790}\NormalTok{, }\DecValTok{1970}\NormalTok{, }\DecValTok{10}\NormalTok{)}
\ControlFlowTok{for}\NormalTok{(i }\ControlFlowTok{in}\NormalTok{ year)\{}
  \KeywordTok{print}\NormalTok{(i)}
\NormalTok{\}}
\end{Highlighting}
\end{Shaded}

\begin{verbatim}
## [1] 1790
## [1] 1800
## [1] 1810
## [1] 1820
## [1] 1830
## [1] 1840
## [1] 1850
## [1] 1860
## [1] 1870
## [1] 1880
## [1] 1890
## [1] 1900
## [1] 1910
## [1] 1920
## [1] 1930
## [1] 1940
## [1] 1950
## [1] 1960
## [1] 1970
\end{verbatim}

\end{frame}

\begin{frame}[fragile]{Third use with a vector of text strings}

\begin{Shaded}
\begin{Highlighting}[]
\NormalTok{instructors <-}\StringTok{ }\KeywordTok{c}\NormalTok{(}\StringTok{"Arushi"}\NormalTok{, }\StringTok{"Cameron"}\NormalTok{, }\StringTok{"Duy"}\NormalTok{, }
                 \StringTok{"Mitch"}\NormalTok{, }\StringTok{"Zack"}\NormalTok{)}
\ControlFlowTok{for}\NormalTok{(i }\ControlFlowTok{in}\NormalTok{ instructors)\{}
  \KeywordTok{print}\NormalTok{(}\KeywordTok{paste}\NormalTok{(i,}\StringTok{"is an instructor."}\NormalTok{))}
\NormalTok{\}}
\end{Highlighting}
\end{Shaded}

\begin{verbatim}
## [1] "Arushi is an instructor."
## [1] "Cameron is an instructor."
## [1] "Duy is an instructor."
## [1] "Mitch is an instructor."
## [1] "Zack is an instructor."
\end{verbatim}

\end{frame}

\begin{frame}[fragile]{An example with \texttt{for}, \texttt{if}, and
\texttt{else}}

\begin{itemize}
\tightlist
\item
  For each number \texttt{i} in 1:10, output whether \texttt{i} is even.
\item
  To do this, we will use \texttt{if} statements within a \texttt{for}
  loop.
\item
  First, let's introduce \texttt{\%\%} and \texttt{paste()}:
\end{itemize}

\begin{Shaded}
\begin{Highlighting}[]
\DecValTok{7} \OperatorTok\StringTok{ }\DecValTok{5}
\end{Highlighting}
\end{Shaded}

\begin{verbatim}
## [1] 2
\end{verbatim}

\begin{Shaded}
\begin{Highlighting}[]
\NormalTok{remainder <-}\StringTok{ }\DecValTok{7} \OperatorTok\StringTok{ }\DecValTok{5}
\KeywordTok{paste}\NormalTok{(remainder,}\StringTok{"is the remainder"}\NormalTok{)}
\end{Highlighting}
\end{Shaded}

\begin{verbatim}
## [1] "2 is the remainder"
\end{verbatim}

\end{frame}

\begin{frame}[fragile]{An example with \texttt{for}, \texttt{if}, and
\texttt{else}}

Step 1:

\begin{itemize}
\tightlist
\item
  Write the \texttt{if} statements needed to test whether a number
  \texttt{i} is even
\item
  If it is even, output the number and ``is even.''
\item
  If it is odd, output the number and ``is odd''
\end{itemize}

\end{frame}

\begin{frame}[fragile]{An example with \texttt{for}, \texttt{if}, and
\texttt{else}}

\begin{Shaded}
\begin{Highlighting}[]
  \ControlFlowTok{if}\NormalTok{(i }\OperatorTok\StringTok{ }\DecValTok{2} \OperatorTok{==}\StringTok{ }\DecValTok{0}\NormalTok{)\{ }
    \KeywordTok{print}\NormalTok{(}\KeywordTok{paste}\NormalTok{(i,}\StringTok{"is even."}\NormalTok{))}
\NormalTok{  \} }\ControlFlowTok{else}\NormalTok{ \{}
    \KeywordTok{print}\NormalTok{(}\KeywordTok{paste}\NormalTok{(i,}\StringTok{"is odd."}\NormalTok{))}
\NormalTok{  \} }
\end{Highlighting}
\end{Shaded}

\begin{itemize}
\tightlist
\item
  Now do Step 2: add the \texttt{for} loop so that the commands above
  are performed over all values in 1:10
\end{itemize}

\end{frame}

\begin{frame}[fragile]{An example with \texttt{for}, \texttt{if}, and
\texttt{else}}

\begin{Shaded}
\begin{Highlighting}[]
\ControlFlowTok{for}\NormalTok{(i }\ControlFlowTok{in} \DecValTok{1}\OperatorTok{:}\DecValTok{10}\NormalTok{)\{}
  \ControlFlowTok{if}\NormalTok{(i }\OperatorTok\StringTok{ }\DecValTok{2} \OperatorTok{==}\StringTok{ }\DecValTok{0}\NormalTok{)\{ }
    \KeywordTok{print}\NormalTok{(}\KeywordTok{paste}\NormalTok{(i,}\StringTok{"is even."}\NormalTok{))}
\NormalTok{  \} }\ControlFlowTok{else}\NormalTok{ \{}
    \KeywordTok{print}\NormalTok{(}\KeywordTok{paste}\NormalTok{(i,}\StringTok{"is odd."}\NormalTok{))}
\NormalTok{  \} }
\NormalTok{\}}
\end{Highlighting}
\end{Shaded}

\end{frame}

\begin{frame}[fragile]{An example with \texttt{for}, \texttt{if}, and
\texttt{else}}

\begin{Shaded}
\begin{Highlighting}[]
\ControlFlowTok{for}\NormalTok{(i }\ControlFlowTok{in} \DecValTok{1}\OperatorTok{:}\DecValTok{10}\NormalTok{)\{}
  \ControlFlowTok{if}\NormalTok{(i }\OperatorTok\StringTok{ }\DecValTok{2} \OperatorTok{==}\StringTok{ }\DecValTok{0}\NormalTok{)\{ }
    \KeywordTok{print}\NormalTok{(}\KeywordTok{paste}\NormalTok{(i,}\StringTok{"is even."}\NormalTok{))}
\NormalTok{  \} }\ControlFlowTok{else}\NormalTok{ \{}
    \KeywordTok{print}\NormalTok{(}\KeywordTok{paste}\NormalTok{(i,}\StringTok{"is odd."}\NormalTok{))}
\NormalTok{  \} }
\NormalTok{\}}
\end{Highlighting}
\end{Shaded}

\begin{verbatim}
## [1] "1 is odd."
## [1] "2 is even."
## [1] "3 is odd."
## [1] "4 is even."
## [1] "5 is odd."
## [1] "6 is even."
## [1] "7 is odd."
## [1] "8 is even."
## [1] "9 is odd."
## [1] "10 is even."
\end{verbatim}

\end{frame}

\begin{frame}[fragile]{\textbf{\emph{\texttt{while}}} Loops in
\texttt{R}}

\begin{itemize}
\tightlist
\item
  The general form is:
\end{itemize}

\begin{Shaded}
\begin{Highlighting}[]
\ControlFlowTok{while}\NormalTok{(logical test)\{}
\NormalTok{  do some stuff until logical test becomes false}
\NormalTok{\} }
\end{Highlighting}
\end{Shaded}

\begin{itemize}
\tightlist
\item
  For example:
\end{itemize}

\begin{Shaded}
\begin{Highlighting}[]
\ControlFlowTok{while}\NormalTok{(money }\OperatorTok{>}\StringTok{ }\ErrorTok{$}\DecValTok{0}\NormalTok{)\{}
  \KeywordTok{eat}\NormalTok{(food)}
\NormalTok{  money <-}\StringTok{ }\NormalTok{money }\OperatorTok{-}\StringTok{ }\KeywordTok{cost}\NormalTok{(food)}
\NormalTok{\} }
\end{Highlighting}
\end{Shaded}

\begin{itemize}
\tightlist
\item
  When does the loop end? What happens if food is free?
\end{itemize}

\end{frame}

\begin{frame}[fragile]{Example in \texttt{R}}

\begin{Shaded}
\begin{Highlighting}[]
\NormalTok{x <-}\StringTok{ }\DecValTok{0}
\NormalTok{limit <-}\StringTok{ }\DecValTok{10}
\ControlFlowTok{while}\NormalTok{(x }\OperatorTok{<=}\StringTok{ }\NormalTok{limit)\{}
\NormalTok{  x <-}\StringTok{ }\NormalTok{x }\OperatorTok{+}\StringTok{ }\DecValTok{2}
  \KeywordTok{print}\NormalTok{(x)}
\NormalTok{\}}
\end{Highlighting}
\end{Shaded}

\begin{verbatim}
## [1] 2
## [1] 4
## [1] 6
## [1] 8
## [1] 10
## [1] 12
\end{verbatim}

How many iterations does this loop go through?

\end{frame}

\begin{frame}[fragile]{An example with \texttt{while}, \texttt{if}, and
\texttt{else}}

\begin{Shaded}
\begin{Highlighting}[]
\NormalTok{i <-}\StringTok{ }\DecValTok{1}
\ControlFlowTok{while}\NormalTok{(i }\OperatorTok{<=}\StringTok{ }\DecValTok{10}\NormalTok{)\{}
  \ControlFlowTok{if}\NormalTok{(i }\OperatorTok\StringTok{ }\DecValTok{2} \OperatorTok{==}\StringTok{ }\DecValTok{0}\NormalTok{)\{ }
    \KeywordTok{print}\NormalTok{(}\KeywordTok{paste}\NormalTok{(i,}\StringTok{"is even."}\NormalTok{))}
\NormalTok{  \} }\ControlFlowTok{else}\NormalTok{ \{}
    \KeywordTok{print}\NormalTok{(}\KeywordTok{paste}\NormalTok{(i,}\StringTok{"is odd."}\NormalTok{))}
\NormalTok{  \} }
\NormalTok{  i <-}\StringTok{ }\NormalTok{i }\OperatorTok{+}\StringTok{ }\DecValTok{1}
\NormalTok{\}}
\end{Highlighting}
\end{Shaded}

\begin{verbatim}
## [1] "1 is odd."
## [1] "2 is even."
## [1] "3 is odd."
## [1] "4 is even."
## [1] "5 is odd."
## [1] "6 is even."
## [1] "7 is odd."
## [1] "8 is even."
## [1] "9 is odd."
## [1] "10 is even."
\end{verbatim}

\end{frame}

\begin{frame}[fragile]{Your turn!}

\begin{itemize}
\tightlist
\item
  Work on class20.R and/or class19.R.
\end{itemize}

Here are the commands/operators we covered today:

\begin{itemize}
\tightlist
\item
  \texttt{for,\ if,\ else}
\item
  \texttt{in,\ print,\ sample}
\end{itemize}

\end{frame}

\end{document}
