\documentclass[11pt]{beamer}
\mode<presentation>
%\documentclass[handout,compress]{beamer}
\usepackage{beamerthemedefault}
\usepackage{graphicx}
\usepackage{hyperref}
\usepackage{subfigure}
\usepackage{color}
\usepackage{multicol}
\usepackage{bm} 
\usepackage{tikz}
\usepackage{listliketab}

\usetheme{CambridgeUS}
\makeatletter
\makeatother
\usetikzlibrary{shapes,backgrounds}
\tikzstyle{cblue}=[circle, draw, thin,fill=cyan!20, scale=0.8]
\tikzstyle{qgre}=[rectangle, draw, thin,fill=green!20, scale=0.8]
\tikzstyle{rpath}=[ultra thick, red, opacity=0.4]
\tikzstyle{legend_isps}=[rectangle, rounded corners, thin,
                       fill=gray!20, text=blue, draw]
\usetikzlibrary{decorations.pathreplacing}
\tikzset{text/.default=}
%\tikzset{text/.align=0}
\tikzstyle{every picture}+=[remember picture]
\tikzstyle{na} = [baseline=-.5ex]
\setbeamertemplate{itemize item}{\color{black}$\bullet$}
\setbeamertemplate{itemize subitem}{\color{black}$\bullet$}

\usetikzlibrary{shapes}
\usetikzlibrary{positioning}
\usetikzlibrary{automata}
\usepackage{amsmath,amssymb,amsfonts,amsthm}
\setbeamercovered{invisible}
\newcommand{\red}{\textcolor{red}}
\newcommand{\blue}{\textcolor{blue}}
\newcommand{\purple}{\textcolor{purple}}
\newcommand{\brown}{\textcolor{brown}}
\newcommand{\cyan}{\textcolor{cyan}}
\newcommand{\real}{\ensuremath{\mathbb{R}}}
\newcommand{\y}{\ensuremath{\mathbf{y}}}
\newcommand{\black}{\color{black}}
\newcommand{\btheta}{\boldsymbol{\theta}}
\newcommand{\green}{\color{green}}
\newcommand{\word}[1]{\green{\textit{#1}\ }\black}
\newcommand{\lb}{\linebreak}
\newtheorem{com}{Comment}
\newtheorem{lem} {Lemma}
\newtheorem{prop}{Proposition}
\newtheorem{thm}{Theorem}
\newtheorem{defn}{Definition}
\newtheorem{cor}{Corollary}
\newtheorem{obs}{Observation}
\setcounter{tocdepth}{1}

\definecolor{UBCblue}{rgb}{0.04706, 0.13725, 0.26667}
\definecolor{UBCgray}{rgb}{0.3686, 0.5255, 0.6235}
\colorlet{verylightgray}{gray!10}
\setbeamercolor{palette primary}{bg=UBCblue,fg=white}
\setbeamercolor{palette secondary}{bg=darkgray,fg=white}
\setbeamercolor{palette tertiary}{bg=UBCblue,fg=white}
\setbeamercolor{palette quaternary}{bg=UBCblue,fg=white}
\setbeamercolor{structure}{fg=UBCblue} % itemize, enumerate, etc
\setbeamercolor{section in toc}{fg=UBCblue} % TOC sections
\setbeamercolor{subsection in head/foot}{bg=darkgray,fg=white}
\setbeamercolor{frametitle}{fg=UBCblue}
\setbeamercolor{title}{fg=UBCblue, bg=verylightgray}

\title[Class 5]{Introduction to Social Data Analytics \\
\bigskip Class 5}
\author[Kaushik]{Arushi Kaushik}
\institute[UCSD]{arkaushi@ucsd.edu}
\date{}

% update title block above


\begin{document}

\frame{\titlepage}

\begin{frame}
 \frametitle{Today's Structure}
\begin{itemize} \itemsep1em
%\item Quickly review the quiz 
\item Plotting in Excel 
	\begin{itemize}
	\item Download and open class5.xlsx if you haven't already.
	\end{itemize}
\item Work in pairs and help each other as needed.
\end{itemize}
\end{frame}

\begin{frame}
\frametitle{Today's Learning Objectives}
From Pre Class 5 Exercise: 
\begin{itemize}
	\item nested IF statements, IFNA, COUNTIFS, status bar sum, insert scatterplot, edit titles and axis labels
\end{itemize} \pause
\bigskip
After today, you should be able to:
\begin{itemize}
	\item Create the following plots in Excel: scatter, histogram, bar, pie
	\item Add elements to plots including titles, axis labels, trendlines, etc.
	\item Adjust axis ranges, bin sizes, and colors
\end{itemize} % \pause \bigskip 
%Please download and open class5.xlsx if you haven't already.
\end{frame}

\begin{frame}
\frametitle{The basics of Excel plots}
\begin{center}
	\includegraphics[width = \textwidth]{images/excel1.png} 
\end{center}
\begin{itemize}
	\item First, select relevant data (e.g. columns B and C) \pause
	\item Second, via the Insert tab, choose the plot category you want to create (e.g. Scatter)
\end{itemize}
\end{frame}

\begin{frame}
\frametitle{The basics of Excel plots}
\begin{center}
	\includegraphics[width = 0.8 \textwidth]{images/excel2.png} 
\end{center}
\begin{itemize}
	\item Third, hover over the plot type and check the preview. Click if it's the plot type you want to create.
\end{itemize}
\end{frame}

\begin{frame}
\frametitle{Adding elements to a plot}
\begin{center}
	\includegraphics[width = \textwidth]{images/excel3.png} 
\end{center}
\begin{itemize}
	\item Select the plot and click the plus near the top right corner \pause
	\item Add axis titles and a trendline 
\end{itemize}
\end{frame}

\begin{frame}
\frametitle{Editing plot label text}
\begin{center}
	\includegraphics[width = \textwidth]{images/excel4.png} 
\end{center}
\begin{itemize}
	\item Click the plot title and axis titles to edit the text 
\end{itemize}
\end{frame}

\begin{frame}
\frametitle{Changing the trendline options}
\begin{center}
	\includegraphics[width = 0.7 \textwidth]{images/excel5.png} 
\end{center}
\begin{itemize}
	\item Right-click the trendline and select ``Format Trendline" \pause
	\item Navigate to the rightmost sub-menu (looks like a barplot) \pause
	\item Select ``Display Equation on chart" and ``Display R-squared value"
\end{itemize}
\end{frame}

\begin{frame}
\frametitle{Editing axis scaling}
\begin{center}
	\includegraphics[width = \textwidth]{images/excel6.png} 
\end{center}
\begin{itemize}
	\item Right-click the y-axis and select ``Format Axis" \pause
	\item Set the maximum bound to 100 \pause 
	\item Click the x-axis and set the max bound to 4.0
\end{itemize}
\end{frame}

\begin{frame}
\frametitle{Editing axis scaling}
\begin{center}
	\includegraphics[width = 0.7 \textwidth]{images/excel7.png} 
\end{center}
\begin{itemize}
	\item Towards the bottom of the ``Format Axis" menu, click the ``Number" sub-sub-menu \pause
	\item Change the category to ``Number" and ensure there are two decimal places 
\end{itemize}
\end{frame}

\begin{frame}
\frametitle{Changing colors}
\begin{center}
	\includegraphics[width = 0.7 \textwidth]{images/excel8.png} 
\end{center}
\begin{itemize}
	\item Right-click a point in the plot and select ``Format Data Series" \pause
	\item Navigate to the leftmost sub-menu (paint can), and select ``Marker" \pause
	\item Change the following: type, size, fill color, border color
\end{itemize}
\end{frame}

\begin{frame}
\frametitle{The rest of today's class}
\begin{itemize} \itemsep1em
	\item Try creating a histogram, barplot, and pie chart \pause
	\item Each sheet contains the data necessary for each plot \pause 
	\item An example of each plot is shown on the first sheet (``Data") \pause
	\item Help your partners, and we'll finish the rest of the plots on Friday.
\end{itemize}
\end{frame}

\begin{frame}
\frametitle{Next class}
\begin{itemize} \itemsep1em
	\item[] Friday we will continue plotting in Excel. 
	\item[] See you then!
\end{itemize}
\end{frame}
\end{document}









