\documentclass[11pt]{beamer}
\mode<presentation>
%\documentclass[handout,compress]{beamer}
\usepackage{beamerthemedefault}
\usepackage{graphicx}
\usepackage{hyperref}
\usepackage{subfigure}
\usepackage{color}
\usepackage{multicol}
\usepackage{bm} 
\usepackage{tikz}
\usepackage{listliketab}

\usetheme{CambridgeUS}
\makeatletter
\makeatother
\usetikzlibrary{shapes,backgrounds}
\tikzstyle{cblue}=[circle, draw, thin,fill=cyan!20, scale=0.8]
\tikzstyle{qgre}=[rectangle, draw, thin,fill=green!20, scale=0.8]
\tikzstyle{rpath}=[ultra thick, red, opacity=0.4]
\tikzstyle{legend_isps}=[rectangle, rounded corners, thin,
                       fill=gray!20, text=blue, draw]
\usetikzlibrary{decorations.pathreplacing}
\tikzset{text/.default=}
%\tikzset{text/.align=0}
\tikzstyle{every picture}+=[remember picture]
\tikzstyle{na} = [baseline=-.5ex]
\setbeamertemplate{items}[default]
\setbeamertemplate{itemize item}{\color{black}$\bullet$}
\setbeamertemplate{itemize subitem}{\color{black}$\bullet$}

\usetikzlibrary{shapes}
\usetikzlibrary{positioning}
\usetikzlibrary{automata}
\usepackage{amsmath,amssymb,amsfonts,amsthm}
\setbeamercovered{invisible}
\newcommand{\red}{\textcolor{red}}
\newcommand{\blue}{\textcolor{blue}}
\newcommand{\purple}{\textcolor{purple}}
\newcommand{\brown}{\textcolor{brown}}
\newcommand{\cyan}{\textcolor{cyan}}
\newcommand{\real}{\ensuremath{\mathbb{R}}}
\newcommand{\y}{\ensuremath{\mathbf{y}}}
\newcommand{\black}{\color{black}}
\newcommand{\btheta}{\boldsymbol{\theta}}
\newcommand{\green}{\color{green}}
\newcommand{\word}[1]{\green{\textit{#1}\ }\black}
\newcommand{\lb}{\linebreak}
\newtheorem{com}{Comment}
\newtheorem{lem} {Lemma}
\newtheorem{prop}{Proposition}
\newtheorem{thm}{Theorem}
\newtheorem{defn}{Definition}
\newtheorem{cor}{Corollary}
\newtheorem{obs}{Observation}
\setcounter{tocdepth}{1}

\definecolor{UBCblue}{rgb}{0.04706, 0.13725, 0.26667}
\definecolor{UBCgray}{rgb}{0.3686, 0.5255, 0.6235}
\colorlet{verylightgray}{gray!10}
\setbeamercolor{palette primary}{bg=UBCblue,fg=white}
\setbeamercolor{palette secondary}{bg=darkgray,fg=white}
\setbeamercolor{palette tertiary}{bg=UBCblue,fg=white}
\setbeamercolor{palette quaternary}{bg=UBCblue,fg=white}
\setbeamercolor{structure}{fg=UBCblue} % itemize, enumerate, etc
\setbeamercolor{section in toc}{fg=UBCblue} % TOC sections
\setbeamercolor{subsection in head/foot}{bg=darkgray,fg=white}
\setbeamercolor{frametitle}{fg=UBCblue}
\setbeamercolor{title}{fg=UBCblue, bg=verylightgray}


\title[Class 4]{Introduction to Social Data Analytics \\
\bigskip Class 4}
\author[Kaushik]{Arushi Kaushik}
\institute[UCSD]{arkaushi@ucsd.edu}
\date{}

% Update title block above
% Add graphics using data from your section's survey. Be sure to explain what kind of variable each plot uses. 

\begin{document}

\frame{\titlepage}

%\begin{frame}
% \frametitle{Announcements}
%\begin{itemize} \itemsep1em
%\item Form teams of two for today
%\end{itemize}
%\end{frame}

\begin{frame}
 \frametitle{Today's Learning Objectives}
 From Pre Class 4 Exercise: 
 \begin{itemize}
 	\item resizing columns, pasting values, using MATCH and VLOOKUP
\end{itemize} \pause
\bigskip
 After today, you should be able to:
\begin{itemize}
\item Classify kinds of variables (numerical, (un)ordered categorical, logical)
\item Identify when a sample contains sampling bias and implications for external validity
\item Use the RAND function to conduct a simple random sample
\end{itemize} \pause \bigskip 
Please download and open class4.xlsx if you haven't already.
\end{frame}

\begin{frame}
 \frametitle{Kinds of Variables}
\begin{itemize}
\item \alert{Numerical} (Quantitative): Represented by a number \pause
\item \alert{Categorical}: Data in categories
\begin{itemize}
\item Unordered: Unordered categories
\item Ordinal: Ordered categories
\end{itemize} \pause
\item \alert{Logical}: True/False variables
\end{itemize}
\end{frame}

\begin{frame}
 \frametitle{Which kinds of variables are the following:}
\alert{Numerical, unordered categorical, ordered categorical, or logical}
\begin{itemize}
\pause
\bigskip
\item Rating of High, Medium, or Low of a student's confidence with Excel \pause
\item Whether or not a student turned in an assignment on time \pause
\item A student's score on an exam \pause
\item A student's college at UCSD
\end{itemize}
\end{frame}

\begin{frame}
 \frametitle{What year are you?}
 \begin{center}
 \includegraphics[width = 0.5\textwidth]{Year} 
  \includegraphics[width = 0.5\textwidth]{major} 
 \end{center}
\end{frame}

\begin{frame}
 \frametitle{Check all software you've used before.}
 \begin{center}
 \includegraphics[width = 0.9\textwidth]{softwares} % use a graphic specific to your class
 \end{center}
\end{frame}

\begin{frame}
 \frametitle{The kind of variable depends on the question}
Classify the following variables:
\begin{itemize}
\item Do you know how to use Excel? \pause
\item On a scale from 1 to 10 how well do you know Excel? \pause
\item Across sections, what \% have used Excel before? \pause
\item Which software package do you know best?
\end{itemize}
\end{frame}

\begin{frame}
 \frametitle{How can we learn from a sample?}
 We are often interested in \alert{population} averages (e.g. poverty rates, election turnout, etc.), but polling the entire population is typically too costly. \\ \pause
 \bigskip We settle for a \alert{sample} and use statistics instead. \\ \pause
 \bigskip
\uncover{How can we learn from a sample?}
\begin{itemize}
\item \alert{External validity}: when relationships found within the sample data apply to the population \pause
\item To be externally valid, the sample must be \alert{representative} of the population \pause
\begin{itemize}
	\item One technique is \alert{simple random sampling} \pause
	\item Difficult in practice - what happens when some of the population cannot be contacted (e.g. no phone or email)? \pause
	\item Non-responses result in \alert{sampling bias}
\end{itemize}
\end{itemize}
\end{frame}

\begin{frame}
\frametitle{Another kind of bias: \alert{response bias}}
Often in surveys, respondents provide answers that do not wholly reflect the truth. \\ \bigskip \pause
Consider how respondents might be inclined to provide biased answers to the following:
\begin{itemize}
	\item How often do you make racist comments? \pause (\alert{Social pressure} to answer a certain way) \pause
	\item How much income do you earn per year? \pause (Too much effort to report exact income $\Rightarrow$ \alert{rounding}) \pause
	\item How likely are you to recommend my company's product to a friend? \pause (Don't want to hurt surveyor's feelings)
\end{itemize}
\end{frame}

\begin{frame}
 \frametitle{Example: wages after graduation}
Universities often ask alumni for information about their careers over time.
\begin{itemize}
\item What is the \alert{population} of interest?\pause
\item What is the \alert{sample}?\pause
\item Any concerns with phone or mail based surveys?\pause
% \item What is the \alert{sampling frame}? \pause
\item Why might the sample \textit{not} be representative (contain sampling bias)? \pause
\item Why might the data contain response bias? 
\end{itemize}
\end{frame}

\begin{frame}
\frametitle{class4.xlsx}
 \begin{center}
	\includegraphics[width = \textwidth]{images/excel1.png} 
\end{center}
\begin{itemize}
	\item What is the unit of analysis? \pause
	\item Use the \alert{RAND()} function to assign a random number (column E) \pause
	\item If the random number $>$ 0.5, they are selected for the random sample (column F) \pause
	\item If the random number $>$ 0.5 but phone = ``N", they are NOT included in the random sample with sampling bias (column G) \pause
	\item Determine the average reported salary amongst those selected (J3) and amongst those selected who answered their phones (J4). 
\end{itemize}
\end{frame}

%\begin{frame}
% \frametitle{General Tips}
% \uncover{Tip 1: Cleaning is the hardest and most important step}
%\begin{itemize}
%\item \alert{Data cleaning}: preparing your data for analysis
%\item What unit of analysis will you use?
%\item How will you deal with \alert{missing values}?
%\end{itemize}
%\medskip\pause
%
%\uncover{Tip 2: Be cognizant of \alert{data types}}
%\begin{itemize}
%\item Numerical, string, factor, or logical?
%\item Is the software treating each variable how you want it to?
%\item Does the number represent a category or a true number? (e.g. income vs. income ranges)
%\end{itemize}
%\end{frame}
%
%\begin{frame}
%\frametitle{More General Tips}
%\uncover{Tip 3: Keep your raw data safe}
%\begin{itemize}
%	\item Work somewhere that won't overwrite your data
%\end{itemize} \pause
%\medskip
%\uncover{Tip 4: Debug and check your work}
%\begin{itemize}
%	\item Google Search is your friend. Likely someone has already answered your question on StackExchange and the like.
%	\item Break down tough problems into a sequence of smaller ones.
%\end{itemize}
%\end{frame}

\begin{frame}
 \frametitle{Next class}
\begin{enumerate}
\item We will be plotting in Excel.
\item Submit your pre class exercise before class.
\end{enumerate}
\end{frame}

\end{document}









