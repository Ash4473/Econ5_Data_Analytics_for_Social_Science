\documentclass[ignorenonframetext,]{beamer}
\setbeamertemplate{caption}[numbered]
\setbeamertemplate{caption label separator}{: }
\setbeamercolor{caption name}{fg=normal text.fg}
\beamertemplatenavigationsymbolsempty
\usepackage{lmodern}
\usepackage{amssymb,amsmath}
\usepackage{ifxetex,ifluatex}
\usepackage{fixltx2e} % provides \textsubscript

\usetheme{CambridgeUS}
\definecolor{UBCblue}{rgb}{0.04706, 0.13725, 0.26667}
\definecolor{UBCgray}{rgb}{0.3686, 0.5255, 0.6235}
\colorlet{verylightgray}{gray!10}
\setbeamercolor{palette primary}{bg=UBCblue,fg=white}
\setbeamercolor{palette secondary}{bg=darkgray,fg=white}
\setbeamercolor{palette tertiary}{bg=UBCblue,fg=white}
\setbeamercolor{palette quaternary}{bg=UBCblue,fg=white}
\setbeamercolor{structure}{fg=UBCblue} % itemize, enumerate, etc
\setbeamercolor{section in toc}{fg=UBCblue} % TOC sections
\setbeamercolor{subsection in head/foot}{bg=darkgray,fg=white}
\setbeamercolor{frametitle}{fg=UBCblue}
\setbeamercolor{title}{fg=UBCblue, bg=verylightgray}
\setbeamertemplate{itemize items}{\color{UBCblue}$\blacktriangleright$}


\ifnum 0\ifxetex 1\fi\ifluatex 1\fi=0 % if pdftex
\usepackage[T1]{fontenc}
\usepackage[utf8]{inputenc}
\else % if luatex or xelatex
\ifxetex
\usepackage{mathspec}
\else
\usepackage{fontspec}
\fi
\defaultfontfeatures{Ligatures=TeX,Scale=MatchLowercase}
\fi
% use upquote if available, for straight quotes in verbatim environments
\IfFileExists{upquote.sty}{\usepackage{upquote}}{}
% use microtype if available
\IfFileExists{microtype.sty}{%
	\usepackage{microtype}
	\UseMicrotypeSet[protrusion]{basicmath} % disable protrusion for tt fonts
}{}
\newif\ifbibliography
\hypersetup{
	pdfauthor={UCSD},
	pdfborder={0 0 0},
	breaklinks=true}
\urlstyle{same}  % don't use monospace font for urls
\usepackage{color}
\usepackage{fancyvrb}
\newcommand{\VerbBar}{|}
\newcommand{\VERB}{\Verb[commandchars=\\\{\}]}
\DefineVerbatimEnvironment{Highlighting}{Verbatim}{commandchars=\\\{\}}
% Add ',fontsize=\small' for more characters per line
\usepackage{framed}
\definecolor{shadecolor}{RGB}{248,248,248}
\newenvironment{Shaded}{\begin{snugshade}}{\end{snugshade}}
\newcommand{\KeywordTok}[1]{\textcolor[rgb]{0.13,0.29,0.53}{\textbf{#1}}}
\newcommand{\DataTypeTok}[1]{\textcolor[rgb]{0.13,0.29,0.53}{#1}}
\newcommand{\DecValTok}[1]{\textcolor[rgb]{0.00,0.00,0.81}{#1}}
\newcommand{\BaseNTok}[1]{\textcolor[rgb]{0.00,0.00,0.81}{#1}}
\newcommand{\FloatTok}[1]{\textcolor[rgb]{0.00,0.00,0.81}{#1}}
\newcommand{\ConstantTok}[1]{\textcolor[rgb]{0.00,0.00,0.00}{#1}}
\newcommand{\CharTok}[1]{\textcolor[rgb]{0.31,0.60,0.02}{#1}}
\newcommand{\SpecialCharTok}[1]{\textcolor[rgb]{0.00,0.00,0.00}{#1}}
\newcommand{\StringTok}[1]{\textcolor[rgb]{0.31,0.60,0.02}{#1}}
\newcommand{\VerbatimStringTok}[1]{\textcolor[rgb]{0.31,0.60,0.02}{#1}}
\newcommand{\SpecialStringTok}[1]{\textcolor[rgb]{0.31,0.60,0.02}{#1}}
\newcommand{\ImportTok}[1]{#1}
\newcommand{\CommentTok}[1]{\textcolor[rgb]{0.56,0.35,0.01}{\textit{#1}}}
\newcommand{\DocumentationTok}[1]{\textcolor[rgb]{0.56,0.35,0.01}{\textbf{\textit{#1}}}}
\newcommand{\AnnotationTok}[1]{\textcolor[rgb]{0.56,0.35,0.01}{\textbf{\textit{#1}}}}
\newcommand{\CommentVarTok}[1]{\textcolor[rgb]{0.56,0.35,0.01}{\textbf{\textit{#1}}}}
\newcommand{\OtherTok}[1]{\textcolor[rgb]{0.56,0.35,0.01}{#1}}
\newcommand{\FunctionTok}[1]{\textcolor[rgb]{0.00,0.00,0.00}{#1}}
\newcommand{\VariableTok}[1]{\textcolor[rgb]{0.00,0.00,0.00}{#1}}
\newcommand{\ControlFlowTok}[1]{\textcolor[rgb]{0.13,0.29,0.53}{\textbf{#1}}}
\newcommand{\OperatorTok}[1]{\textcolor[rgb]{0.81,0.36,0.00}{\textbf{#1}}}
\newcommand{\BuiltInTok}[1]{#1}
\newcommand{\ExtensionTok}[1]{#1}
\newcommand{\PreprocessorTok}[1]{\textcolor[rgb]{0.56,0.35,0.01}{\textit{#1}}}
\newcommand{\AttributeTok}[1]{\textcolor[rgb]{0.77,0.63,0.00}{#1}}
\newcommand{\RegionMarkerTok}[1]{#1}
\newcommand{\InformationTok}[1]{\textcolor[rgb]{0.56,0.35,0.01}{\textbf{\textit{#1}}}}
\newcommand{\WarningTok}[1]{\textcolor[rgb]{0.56,0.35,0.01}{\textbf{\textit{#1}}}}
\newcommand{\AlertTok}[1]{\textcolor[rgb]{0.94,0.16,0.16}{#1}}
\newcommand{\ErrorTok}[1]{\textcolor[rgb]{0.64,0.00,0.00}{\textbf{#1}}}
\newcommand{\NormalTok}[1]{#1}
\usepackage{graphicx,grffile}
\makeatletter
\def\maxwidth{\ifdim\Gin@nat@width>\linewidth\linewidth\else\Gin@nat@width\fi}
\def\maxheight{\ifdim\Gin@nat@height>\textheight0.8\textheight\else\Gin@nat@height\fi}
\makeatother
% Scale images if necessary, so that they will not overflow the page
% margins by default, and it is still possible to overwrite the defaults
% using explicit options in \includegraphics[width, height, ...]{}
\setkeys{Gin}{width=\maxwidth,height=\maxheight,keepaspectratio}

% Prevent slide breaks in the middle of a paragraph:
\widowpenalties 1 10000
\raggedbottom

\AtBeginPart{
	\let\insertpartnumber\relax
	\let\partname\relax
	\frame{\partpage}
}
\AtBeginSection{
	\ifbibliography
	\else
	\let\insertsectionnumber\relax
	\let\sectionname\relax
	\frame{\sectionpage}
	\fi
}
\AtBeginSubsection{
	\let\insertsubsectionnumber\relax
	\let\subsectionname\relax
	\frame{\subsectionpage}
}

\setlength{\parindent}{0pt}
\setlength{\parskip}{6pt plus 2pt minus 1pt}
\setlength{\emergencystretch}{3em}  % prevent overfull lines
\providecommand{\tightlist}{%
	\setlength{\itemsep}{0pt}\setlength{\parskip}{0pt}}
\setcounter{secnumdepth}{0}


\title[Class 17]{Introduction to Social Data Analytics\\
	Class 17}
\author[Kaushik]{Arushi Kaushik}
\institute[UCSD]{arkaushi@ucsd.edu}




\begin{document}
\frame{\titlepage}

\begin{frame}{Today: logic and subsetting in \texttt{R}}

By the end of today's lecture, you should be able to:

\begin{itemize}
\tightlist
\item
  Write logic statements in R and identify the relevant Boolean
  operators
\item
  Generate subsets of data using logic operators and \texttt{\$}
\end{itemize}

Open class17.R if you haven't already.

\end{frame}

\begin{frame}[fragile]{Any questions from the pre class exercise?}

\begin{Shaded}
\begin{Highlighting}[]
\StringTok{"Triton"} \OperatorTok{!=}\StringTok{ "triton"}
\end{Highlighting}
\end{Shaded}

\begin{verbatim}
## [1] TRUE
\end{verbatim}

\begin{Shaded}
\begin{Highlighting}[]
\DecValTok{5}\OperatorTok{<}\DecValTok{3} \OperatorTok{|}\StringTok{ }\OtherTok{FALSE}
\end{Highlighting}
\end{Shaded}

\begin{verbatim}
## [1] FALSE
\end{verbatim}

\begin{Shaded}
\begin{Highlighting}[]
\NormalTok{vec1 <-}\StringTok{ }\KeywordTok{c}\NormalTok{(}\OtherTok{TRUE}\NormalTok{, }\OtherTok{TRUE}\NormalTok{, }\OtherTok{FALSE}\NormalTok{, }\OtherTok{TRUE}\NormalTok{); }\KeywordTok{sum}\NormalTok{(vec1) }\OperatorTok{>}\StringTok{ }\DecValTok{3}
\end{Highlighting}
\end{Shaded}

\begin{verbatim}
## [1] FALSE
\end{verbatim}

\end{frame}

\begin{frame}[fragile]{Creating lists and sequences}

\begin{itemize}
\tightlist
\item
  Suppose you want a list of numbers increasing by 1
\end{itemize}

\begin{Shaded}
\begin{Highlighting}[]
\NormalTok{list1 <-}\StringTok{ }\DecValTok{1}\OperatorTok{:}\DecValTok{5}
\NormalTok{list1}
\end{Highlighting}
\end{Shaded}

\begin{verbatim}
## [1] 1 2 3 4 5
\end{verbatim}

\begin{Shaded}
\begin{Highlighting}[]
\NormalTok{list1 }\OperatorTok{>}\StringTok{ }\DecValTok{3}
\end{Highlighting}
\end{Shaded}

\begin{verbatim}
## [1] FALSE FALSE FALSE  TRUE  TRUE
\end{verbatim}

\begin{Shaded}
\begin{Highlighting}[]
\NormalTok{(list1 }\OperatorTok{>}\StringTok{ }\DecValTok{1}\NormalTok{) }\OperatorTok{&}\StringTok{ }\NormalTok{(list1 }\OperatorTok{<}\StringTok{ }\DecValTok{5}\NormalTok{)}
\end{Highlighting}
\end{Shaded}

\begin{verbatim}
## [1] FALSE  TRUE  TRUE  TRUE FALSE
\end{verbatim}

\end{frame}

\begin{frame}[fragile]{Now suppose you want a list of numbers increasing
by 0.5}

\begin{itemize}
\tightlist
\item
  Use the command \texttt{seq(start,\ end,\ increment)}
\end{itemize}

\begin{Shaded}
\begin{Highlighting}[]
\NormalTok{sequence1 <-}\StringTok{ }\KeywordTok{seq}\NormalTok{(}\DecValTok{1}\NormalTok{,}\DecValTok{5}\NormalTok{,}\FloatTok{0.5}\NormalTok{)}
\NormalTok{sequence1}
\end{Highlighting}
\end{Shaded}

\begin{verbatim}
## [1] 1.0 1.5 2.0 2.5 3.0 3.5 4.0 4.5 5.0
\end{verbatim}

\begin{Shaded}
\begin{Highlighting}[]
\NormalTok{(sequence1 }\OperatorTok{>}\StringTok{ }\DecValTok{1}\NormalTok{) }\OperatorTok{&}\StringTok{ }\NormalTok{(sequence1 }\OperatorTok{<}\StringTok{ }\DecValTok{5}\NormalTok{)}
\end{Highlighting}
\end{Shaded}

\begin{verbatim}
## [1] FALSE  TRUE  TRUE  TRUE  TRUE  TRUE  TRUE  TRUE FALSE
\end{verbatim}

\begin{itemize}
\tightlist
\item
  With your partner, generate a sequence with six elements starting at
  10 and ending at 20.
\end{itemize}

\end{frame}

\begin{frame}[fragile]{Some ``class'' examples}

\begin{Shaded}
\begin{Highlighting}[]
\NormalTok{x =}\StringTok{ }\DecValTok{5}
\KeywordTok{class}\NormalTok{(x)}
\end{Highlighting}
\end{Shaded}

\begin{verbatim}
## [1] "numeric"
\end{verbatim}

\begin{Shaded}
\begin{Highlighting}[]
\NormalTok{y =}\StringTok{ "My name is King Triton."}
\KeywordTok{class}\NormalTok{(y)}
\end{Highlighting}
\end{Shaded}

\begin{verbatim}
## [1] "character"
\end{verbatim}

\begin{Shaded}
\begin{Highlighting}[]
\NormalTok{z =}\StringTok{ }\OtherTok{TRUE}
\KeywordTok{class}\NormalTok{(z)}
\end{Highlighting}
\end{Shaded}

\begin{verbatim}
## [1] "logical"
\end{verbatim}

\end{frame}

\begin{frame}{The \textit{class} function outputs the class of the
object you insert}

\begin{itemize}
\tightlist
\item
  Separate classes cannot in general be combined unless you convert them
  first
\item
  We can \alert{coerce} objects to be recognized as different classes
\end{itemize}

\end{frame}

\begin{frame}[fragile]{Objects: coercion with ``as''}

\begin{Shaded}
\begin{Highlighting}[]
\NormalTok{z.str <-}\StringTok{ }\KeywordTok{c}\NormalTok{(}\StringTok{"1"}\NormalTok{, }\StringTok{"2"}\NormalTok{, }\StringTok{"602"}\NormalTok{)}
\NormalTok{z.num <-}\StringTok{ }\KeywordTok{as.numeric}\NormalTok{(z.str)}
\NormalTok{z.str}
\end{Highlighting}
\end{Shaded}

\begin{verbatim}
## [1] "1"   "2"   "602"
\end{verbatim}

\begin{Shaded}
\begin{Highlighting}[]
\NormalTok{z.num}
\end{Highlighting}
\end{Shaded}

\begin{verbatim}
## [1]   1   2 602
\end{verbatim}

\begin{Shaded}
\begin{Highlighting}[]
\CommentTok{# z.str + c(1,1,1) will give an error}
\NormalTok{z.num }\OperatorTok{+}\StringTok{ }\KeywordTok{c}\NormalTok{(}\DecValTok{1}\NormalTok{,}\DecValTok{1}\NormalTok{,}\DecValTok{1}\NormalTok{)}
\end{Highlighting}
\end{Shaded}

\begin{verbatim}
## [1]   2   3 603
\end{verbatim}

\end{frame}

\begin{frame}[fragile]{More ``as'' functions}

\begin{Shaded}
\begin{Highlighting}[]
\NormalTok{a_num <-}\StringTok{ }\KeywordTok{c}\NormalTok{(}\DecValTok{1}\NormalTok{,}\DecValTok{2}\NormalTok{,}\DecValTok{3}\NormalTok{, }\DecValTok{3498}\NormalTok{)}
\NormalTok{a_str <-}\StringTok{ }\KeywordTok{as.character}\NormalTok{(a_num)}
\NormalTok{a_num}
\end{Highlighting}
\end{Shaded}

\begin{verbatim}
## [1]    1    2    3 3498
\end{verbatim}

\begin{Shaded}
\begin{Highlighting}[]
\NormalTok{a_str}
\end{Highlighting}
\end{Shaded}

\begin{verbatim}
## [1] "1"    "2"    "3"    "3498"
\end{verbatim}

\begin{Shaded}
\begin{Highlighting}[]
\NormalTok{a_num }\OperatorTok{+}\StringTok{ }\KeywordTok{c}\NormalTok{(}\DecValTok{1}\NormalTok{,}\DecValTok{1}\NormalTok{,}\DecValTok{1}\NormalTok{,}\DecValTok{1}\NormalTok{)}
\end{Highlighting}
\end{Shaded}

\begin{verbatim}
## [1]    2    3    4 3499
\end{verbatim}

\begin{Shaded}
\begin{Highlighting}[]
\CommentTok{# a_str + c(1,1,1,1) will give an error}
\end{Highlighting}
\end{Shaded}

\end{frame}

\begin{frame}[fragile]{But, ``is'' it what you think?}

\begin{Shaded}
\begin{Highlighting}[]
\KeywordTok{is.character}\NormalTok{(z.num)}
\end{Highlighting}
\end{Shaded}

\begin{verbatim}
## [1] FALSE
\end{verbatim}

\begin{Shaded}
\begin{Highlighting}[]
\KeywordTok{is.character}\NormalTok{(z.str)}
\end{Highlighting}
\end{Shaded}

\begin{verbatim}
## [1] TRUE
\end{verbatim}

\begin{Shaded}
\begin{Highlighting}[]
\KeywordTok{is.numeric}\NormalTok{(a_num)}
\end{Highlighting}
\end{Shaded}

\begin{verbatim}
## [1] TRUE
\end{verbatim}

\begin{Shaded}
\begin{Highlighting}[]
\KeywordTok{is.numeric}\NormalTok{(a_str)}
\end{Highlighting}
\end{Shaded}

\begin{verbatim}
## [1] FALSE
\end{verbatim}

\end{frame}

\begin{frame}[fragile]{How can we tell if something \texttt{isTRUE}?}

\begin{Shaded}
\begin{Highlighting}[]
\KeywordTok{isTRUE}\NormalTok{(}\OtherTok{TRUE}\NormalTok{)}
\end{Highlighting}
\end{Shaded}

\begin{verbatim}
## [1] TRUE
\end{verbatim}

\begin{Shaded}
\begin{Highlighting}[]
\KeywordTok{isTRUE}\NormalTok{(}\OtherTok{FALSE}\NormalTok{)}
\end{Highlighting}
\end{Shaded}

\begin{verbatim}
## [1] FALSE
\end{verbatim}

\begin{Shaded}
\begin{Highlighting}[]
\KeywordTok{isTRUE}\NormalTok{(T)}
\end{Highlighting}
\end{Shaded}

\begin{verbatim}
## [1] TRUE
\end{verbatim}

\begin{Shaded}
\begin{Highlighting}[]
\KeywordTok{isTRUE}\NormalTok{(}\DecValTok{1}\NormalTok{)}
\end{Highlighting}
\end{Shaded}

\begin{verbatim}
## [1] FALSE
\end{verbatim}

\end{frame}

\begin{frame}[fragile]{But be careful\ldots{}}

\begin{Shaded}
\begin{Highlighting}[]
\KeywordTok{as.integer}\NormalTok{(}\OtherTok{TRUE}\NormalTok{)}
\end{Highlighting}
\end{Shaded}

\begin{verbatim}
## [1] 1
\end{verbatim}

\begin{Shaded}
\begin{Highlighting}[]
\NormalTok{T <-}\StringTok{ }\DecValTok{1}
\KeywordTok{isTRUE}\NormalTok{(T)}
\end{Highlighting}
\end{Shaded}

\begin{verbatim}
## [1] FALSE
\end{verbatim}

\begin{itemize}
\tightlist
\item
  T is a predefined ``global'' that is equivalent to TRUE, but can be
  changed.
\end{itemize}

\end{frame}

\begin{frame}[fragile]{Also\ldots{}}

\begin{Shaded}
\begin{Highlighting}[]
\NormalTok{logic <-}\StringTok{ }\KeywordTok{c}\NormalTok{(}\OtherTok{TRUE}\NormalTok{, }\OtherTok{TRUE}\NormalTok{, }\OtherTok{FALSE}\NormalTok{, }\OtherTok{TRUE}\NormalTok{)}
\KeywordTok{mean}\NormalTok{(logic)}
\end{Highlighting}
\end{Shaded}

\begin{verbatim}
## [1] 0.75
\end{verbatim}

\begin{Shaded}
\begin{Highlighting}[]
\KeywordTok{sum}\NormalTok{(logic)}
\end{Highlighting}
\end{Shaded}

\begin{verbatim}
## [1] 3
\end{verbatim}

\end{frame}

\begin{frame}[fragile]{Coercion}

\begin{itemize}
\tightlist
\item
  Coercion is when you call a variable of the wrong type and R tries to
  fix it for you.
\end{itemize}

\begin{Shaded}
\begin{Highlighting}[]
\NormalTok{vec2 <-}\StringTok{ }\KeywordTok{c}\NormalTok{(}\DecValTok{1}\NormalTok{,}\DecValTok{2}\NormalTok{,}\DecValTok{3}\NormalTok{)}
\NormalTok{vec2}
\end{Highlighting}
\end{Shaded}

\begin{verbatim}
## [1] 1 2 3
\end{verbatim}

\begin{Shaded}
\begin{Highlighting}[]
\KeywordTok{typeof}\NormalTok{(vec2)}
\end{Highlighting}
\end{Shaded}

\begin{verbatim}
## [1] "double"
\end{verbatim}

\begin{Shaded}
\begin{Highlighting}[]
\KeywordTok{class}\NormalTok{(vec2)}
\end{Highlighting}
\end{Shaded}

\begin{verbatim}
## [1] "numeric"
\end{verbatim}

\end{frame}

\begin{frame}[fragile]{But \ldots{}}

\begin{Shaded}
\begin{Highlighting}[]
\NormalTok{vec2[}\DecValTok{1}\NormalTok{] =}\StringTok{ "text"}
\KeywordTok{class}\NormalTok{(vec2)}
\end{Highlighting}
\end{Shaded}

\begin{verbatim}
## [1] "character"
\end{verbatim}

\begin{Shaded}
\begin{Highlighting}[]
\NormalTok{vec2}
\end{Highlighting}
\end{Shaded}

\begin{verbatim}
## [1] "text" "2"    "3"
\end{verbatim}

\begin{Shaded}
\begin{Highlighting}[]
\CommentTok{# or}
\OtherTok{TRUE} \OperatorTok{==}\StringTok{ "TRUE"}
\end{Highlighting}
\end{Shaded}

\begin{verbatim}
## [1] TRUE
\end{verbatim}

\begin{itemize}
\tightlist
\item
  Both are examples of coercion.
\end{itemize}

\end{frame}

\begin{frame}[fragile]{Subsetting}

\begin{itemize}
\tightlist
\item
  \texttt{{[}{]}} is for subsets
\item
  \texttt{\$} is for extracting by name (so what you want must be named
  to be used\ldots{})
\item
  What does \texttt{resume\$race{[}1:5{]}} give us?
\item
  What does \texttt{resume{[},"race"{]}} give us?
\item
  What does \texttt{mean(resume\$call{[}resume\$race\ ==\ "black"{]})}
  give us?
\end{itemize}

\end{frame}

\begin{frame}[fragile]{Using subset()}

\begin{itemize}
\tightlist
\item
  \texttt{R} has a function that can help generate subsets of dataframes
\end{itemize}

\begin{Shaded}
\begin{Highlighting}[]
\NormalTok{black <-}\StringTok{ }\KeywordTok{subset}\NormalTok{(resume, }\DataTypeTok{select =} \KeywordTok{c}\NormalTok{(}\StringTok{"firstname"}\NormalTok{, }\StringTok{"sex"}\NormalTok{, }
        \StringTok{"call"}\NormalTok{), }\DataTypeTok{subset =}\NormalTok{ (}\StringTok{race} \OperatorTok{==}\StringTok{ "black"}\NormalTok{))}
\end{Highlighting}
\end{Shaded}

\end{frame}

\begin{frame}[fragile]{Here are the commands/operators we covered
today:}

\begin{itemize}
\tightlist
\item
  \texttt{\&,\ \textbar{},\ !,\ ==}
\item
  \texttt{:,\ seq()}
\item
  \texttt{class(),\ typeof()}
\item
  \texttt{as.numeric(),\ as.character()}
\item
  \texttt{is.numeric(),\ is.character(),\ isTRUE()}
\item
  \texttt{subset()}
\end{itemize}

\end{frame}

\end{document}
