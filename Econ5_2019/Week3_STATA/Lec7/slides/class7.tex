\documentclass[11pt]{beamer}
\mode<presentation>
%\documentclass[handout,compress]{beamer}
\usepackage{beamerthemedefault}
\usepackage{graphicx}
\usepackage{hyperref}
\usepackage{subfigure}
\usepackage{color}
\usepackage{multicol}
\usepackage{bm} 
\usepackage{tikz}
\usepackage{listliketab}
\usepackage{verbatim}

\usetheme{CambridgeUS}
\makeatletter
\makeatother
\usetikzlibrary{shapes,backgrounds}
\tikzstyle{cblue}=[circle, draw, thin,fill=cyan!20, scale=0.8]
\tikzstyle{qgre}=[rectangle, draw, thin,fill=green!20, scale=0.8]
\tikzstyle{rpath}=[ultra thick, red, opacity=0.4]
\tikzstyle{legend_isps}=[rectangle, rounded corners, thin,
                       fill=gray!20, text=blue, draw]
\usetikzlibrary{decorations.pathreplacing}
\tikzset{text/.default=}
%\tikzset{text/.align=0}
\tikzstyle{every picture}+=[remember picture]
\tikzstyle{na} = [baseline=-.5ex]
\setbeamertemplate{itemize item}{\color{black}$\bullet$}
\setbeamertemplate{itemize subitem}{\color{black}$\bullet$}

\usetikzlibrary{shapes}
\usetikzlibrary{positioning}
\usetikzlibrary{automata}
\usepackage{amsmath,amssymb,amsfonts,amsthm}
\setbeamercovered{invisible} %% <--- I ADDED THIS
\newcommand{\red}{\textcolor{red}}
\newcommand{\blue}{\textcolor{blue}}
\newcommand{\purple}{\textcolor{purple}}
\newcommand{\brown}{\textcolor{brown}}
\newcommand{\cyan}{\textcolor{cyan}}
\newcommand{\real}{\ensuremath{\mathbb{R}}}
\newcommand{\y}{\ensuremath{\mathbf{y}}}
\newcommand{\black}{\color{black}}
\newcommand{\btheta}{\boldsymbol{\theta}}
\newcommand{\green}{\color{green}}
\newcommand{\word}[1]{\green{\textit{#1}\ }\black}
\newcommand{\lb}{\linebreak}
\newtheorem{com}{Comment}
\newtheorem{lem} {Lemma}
\newtheorem{prop}{Proposition}
\newtheorem{thm}{Theorem}
\newtheorem{defn}{Definition}
\newtheorem{cor}{Corollary}
\newtheorem{obs}{Observation}
\setcounter{tocdepth}{1}

\definecolor{UBCblue}{rgb}{0.04706, 0.13725, 0.26667}
\definecolor{UBCgray}{rgb}{0.3686, 0.5255, 0.6235}
\colorlet{verylightgray}{gray!10}
\setbeamercolor{palette primary}{bg=UBCblue,fg=white}
\setbeamercolor{palette secondary}{bg=darkgray,fg=white}
\setbeamercolor{palette tertiary}{bg=UBCblue,fg=white}
\setbeamercolor{palette quaternary}{bg=UBCblue,fg=white}
\setbeamercolor{structure}{fg=UBCblue} % itemize, enumerate, etc
\setbeamercolor{section in toc}{fg=UBCblue} % TOC sections
\setbeamercolor{subsection in head/foot}{bg=darkgray,fg=white}
\setbeamercolor{frametitle}{fg=UBCblue}
\setbeamercolor{title}{fg=UBCblue, bg=verylightgray}

\title[Class 7]{Introduction to Social Data Analytics \\
	\bigskip Class 7}
\author[Kaushik]{Arushi Kaushik}
\institute[UCSD]{arkaushi@ucsd.edu}
\date{}

% update title block above

\begin{document}

\frame{\titlepage}

%\begin{frame}
% \frametitle{Housekeeping}
%\begin{itemize} \itemsep1em
%\item Did anyone have trouble submitting problem set 1?
%\item Problem set 2 is posted and due in a week
%\item Quiz 2 is due Wednesday at 9AM.
%\end{itemize}
%\end{frame}

\begin{frame}
\frametitle{Today: Stata}
By the end of today's lecture, you should be able to:
\begin{itemize} \itemsep1em
\item Create, edit, and save Do-files and log files
\item Find more information on commands so you can teach yourself
\item Load data and conduct some of the most common Stata commands
\end{itemize}
\end{frame}

\begin{frame}
\frametitle{Open your Do-file from the pre class 7 exercise}
So far you should have three commands listed:
\begin{itemize} \itemsep1em
	\item \texttt{clear}
	\item \texttt{import excel}
	\item \texttt{summarize}
\end{itemize} \pause \bigskip
Add the new commands that we go over in class to your Do-file.
\end{frame}

\begin{frame}
\frametitle{But first: why do we use Do-files?}
...instead of typing everything in the Command window? \\ \pause \bigskip
\alert{Reproducibility}. \pause \bigskip
\begin{itemize} 
	\item Anyone anywhere with the same data and Do-file can produce the same results. 
	\item Big time savings when repeating analysis of similar data.
\end{itemize}
\end{frame}

\begin{frame}
\frametitle{Command: \texttt{help}}
Syntax: \texttt{help \textit{command}} \\ \medskip
e.g. \texttt{help summarize}
\begin{center}
	\includegraphics[height = 0.75\textheight]{images/help.png}
\end{center}
\end{frame}

\begin{frame}
\frametitle{Command: \texttt{cd} (change directory)}
Syntax: \texttt{cd \textit{path}} \\ \medskip
e.g. \texttt{cd "C:\textbackslash{}User\textbackslash{}Owner\textbackslash{}Documents\textbackslash{}Econ 5\textbackslash{}Class 7"} \bigskip \pause
\begin{itemize} 
	\item Put this at the beginning of your Do-file 
	\item Change your directory to wherever you want to access and save files
\end{itemize}
\end{frame}

\begin{frame}
\frametitle{Command: \texttt{log}}
Syntax: \texttt{log using \textit{filename}, [\textit{options}]} \\ \medskip
e.g. \texttt{log using logfile.txt, text replace} \bigskip \pause
\begin{itemize} \itemsep1em
	\item The log file will record your session - everything you type and the output \pause
	\item This is useful for:
	\begin{itemize}
		\item Tracking what you do
		\item Viewing particularly long output
		\item Sending output to collaborators \pause
	\end{itemize}
	\item At the end of the Do-file, you can stop logging by adding the command \texttt{log close}
\end{itemize}
\end{frame}

\begin{frame}
\frametitle{Commands: \texttt{\underline{d}escribe}, \texttt{\underline{su}mmarize}}
These commands provide information about variables.
\begin{itemize} \itemsep1em
	\item \texttt{desc}: provides a summary of all variables in memory 
	\item \texttt{sum}: produces summary statistics of all variables in memory
\end{itemize} \pause \bigskip
You can use these commands for specific variables instead of every variable in memory. Try:
\begin{itemize} 
	\item \texttt{desc id}
	\item \texttt{sum class adult}
	\item \texttt{sum id - adult survive}
\end{itemize} \pause
What percent of Titanic passengers survived?
\end{frame}

\begin{frame}
\frametitle{Command: \texttt{\underline{tab}ulate}}
Syntax: \texttt{tab \textit{varlist}} \\ \medskip
This command determines the frequency that a variable takes each value. \pause Which of the following variable types is \texttt{tab} useful for?
\begin{itemize}
	\item numerical
	\item categorical
	\item logical
\end{itemize} \pause \bigskip
Try the following:
\begin{itemize}
	\item \texttt{tab male}
	\item \texttt{tab male survive}
	\item \texttt{tab male survive, row}	
\end{itemize} \pause \bigskip
Were men or women more likely to survive the sinking of the Titanic?
\end{frame}

\begin{frame}
\frametitle{Command: \texttt{set more off}}
Suppose you run a command that produces very long output, like \texttt{tab id} (try it!). This command enables you to view all output without having to press enter repeatedly. \\ \pause \bigskip
You can make this change permanent with \texttt{set more off, permanent}
\end{frame}

\begin{frame}
\frametitle{Commands: \texttt{\underline{g}enerate}, \texttt{replace}}
Syntax: \texttt{gen \textit{newvar} = \textit{exp}} \\ \medskip
Syntax: \texttt{replace \textit{oldvar} = \textit{exp}} \\ \bigskip
These commands are used to create variables. \pause Try it:
\begin{itemize}
	\item \texttt{gen notmale = 0}
	\item \texttt{replace notmale = 1 if male == 0}
\end{itemize} \bigskip \pause
Notice the single versus double equals signs. 
\begin{itemize}
	\item \texttt{=} assigns values to a variable
	\item \texttt{==} tests if a variable takes certain values (a logical test)
\end{itemize}
\end{frame}

\begin{frame}
\frametitle{Commands: \texttt{\underline{ren}ame}, \texttt{\underline{la}bel \underline{var}iable}}
Syntax: \texttt{rename \textit{old\_varname} \textit{new\_varname}} \\ \medskip
\pause Rename the variable \texttt{notmale} as \texttt{female}:
\begin{itemize}
	\item \texttt{rename notmale female}
\end{itemize} \bigskip \bigskip \pause

Labeling variables is useful when the name of the variable doesn't fully communicate what the variable represents. \\ \medskip
Syntax: \texttt{label variable \textit{varname} \textit{"label"}} \\ \pause \medskip
Try it:
\begin{itemize}
	\item \texttt{label var female "0 if male, 1 otherwise"}
\end{itemize} \bigskip \bigskip \pause
\end{frame}

\begin{frame}
\frametitle{Commands: \texttt{save}, \texttt{use}}
Stata has its own data file type (.dta). We save and load data in this format with these commands. \pause Try:
\begin{itemize}
	\item \texttt{save titanic, replace}
	\item \texttt{use titanic, clear}
\end{itemize} \bigskip \pause
\texttt{replace} is necessary to overwrite existing files. \pause \\ \bigskip
\texttt{clear} is necessary to load new data when existing data is present. 
\end{frame}

\begin{frame}
\frametitle{Commands introduced in this class:}
\begin{columns}
	\begin{column}{0.5\textwidth}
		\begin{itemize}
			\item \texttt{clear}
			\item \texttt{import excel}
			\item \texttt{help}
			\item \texttt{cd}
			\item \texttt{log using \textit{filename.txt}, text replace}
			\item \texttt{log close}
			\item \texttt{describe}
			\item \texttt{summarize}
		\end{itemize}
	\end{column}
	\begin{column}{0.5\textwidth}
		\begin{itemize}
			\item \texttt{tabulate}
			\item \texttt{set more off}
			\item \texttt{generate}
			\item \texttt{replace}
			\item \texttt{rename}
			\item \texttt{label variable}
			\item \texttt{save  \textit{filename}, replace}
			\item \texttt{use \textit{filename}, clear}
		\end{itemize}
	\end{column}
\end{columns}
\end{frame}

\end{document}









