\documentclass[11pt]{beamer}
\mode<presentation>
%\documentclass[handout,compress]{beamer}
\usepackage{beamerthemedefault}
\usepackage{graphicx}
\usepackage{hyperref}
\usepackage{subfigure}
\usepackage{color}
\usepackage{multicol}
\usepackage{bm} 
\usepackage{tikz}
\usepackage{listliketab}

\usetheme{CambridgeUS}
\makeatletter
\makeatother
\usetikzlibrary{shapes,backgrounds}
\tikzstyle{cblue}=[circle, draw, thin,fill=cyan!20, scale=0.8]
\tikzstyle{qgre}=[rectangle, draw, thin,fill=green!20, scale=0.8]
\tikzstyle{rpath}=[ultra thick, red, opacity=0.4]
\tikzstyle{legend_isps}=[rectangle, rounded corners, thin,
fill=gray!20, text=blue, draw]
\usetikzlibrary{decorations.pathreplacing}
\tikzset{text/.default=}
%\tikzset{text/.align=0}
\tikzstyle{every picture}+=[remember picture]
\tikzstyle{na} = [baseline=-.5ex]
\setbeamertemplate{itemize item}{\color{black}$\bullet$}
\setbeamertemplate{itemize subitem}{\color{black}$\bullet$}

\usetikzlibrary{shapes}
\usetikzlibrary{positioning}
\usetikzlibrary{automata}
\usepackage{amsmath,amssymb,amsfonts,amsthm}
\setbeamercovered{invisible} %% <--- I ADDED THIS
\newcommand{\red}{\textcolor{red}}
\newcommand{\blue}{\textcolor{blue}}
\newcommand{\purple}{\textcolor{purple}}
\newcommand{\brown}{\textcolor{brown}}
\newcommand{\cyan}{\textcolor{cyan}}
\newcommand{\real}{\ensuremath{\mathbb{R}}}
\newcommand{\y}{\ensuremath{\mathbf{y}}}
\newcommand{\black}{\color{black}}
\newcommand{\btheta}{\boldsymbol{\theta}}
\newcommand{\green}{\color{green}}
\newcommand{\word}[1]{\green{\textit{#1}\ }\black}
\newcommand{\lb}{\linebreak}
\newtheorem{com}{Comment}
\newtheorem{lem} {Lemma}
\newtheorem{prop}{Proposition}
\newtheorem{thm}{Theorem}
\newtheorem{defn}{Definition}
\newtheorem{cor}{Corollary}
\newtheorem{obs}{Observation}
\setcounter{tocdepth}{1}

\definecolor{UBCblue}{rgb}{0.04706, 0.13725, 0.26667}
\definecolor{UBCgray}{rgb}{0.3686, 0.5255, 0.6235}
\colorlet{verylightgray}{gray!10}
\setbeamercolor{palette primary}{bg=UBCblue,fg=white}
\setbeamercolor{palette secondary}{bg=darkgray,fg=white}
\setbeamercolor{palette tertiary}{bg=UBCblue,fg=white}
\setbeamercolor{palette quaternary}{bg=UBCblue,fg=white}
\setbeamercolor{structure}{fg=UBCblue} % itemize, enumerate, etc
\setbeamercolor{section in toc}{fg=UBCblue} % TOC sections
\setbeamercolor{subsection in head/foot}{bg=darkgray,fg=white}
\setbeamercolor{frametitle}{fg=UBCblue}
\setbeamercolor{title}{fg=UBCblue, bg=verylightgray}

\title[Class 8]{Introduction to Social Data Analytics \\
	\bigskip Class 8}
\author[]{}
\institute[UCSD]{}
\date{}

% update title block above

% to include in this lecture:
%\texttt{\underline{br}owse}
%Sometimes it's helpful to view our data to examine the structure. We can view our data with \texttt{\underline{br}owse} (try it!). \\ \pause \medskip
%Also try: \texttt{br male female}
% logical testing strings versus numeric, use d to check
% missing values



\begin{document}

\frame{\titlepage}

\begin{frame}
\frametitle{Today: Continue with Stata}
By the end of today's lecture, you should be able to:
\begin{itemize} \itemsep1em
	\item Identify variable types and recall best practices when creating variables
	\item Assign values to variables using functions and logical operators/statements
	\item Sort data and assign values to variables by group designation 
\end{itemize}
\end{frame}

\begin{frame}
\frametitle{Today's Structure}
\begin{itemize} \itemsep1em
	\item Load titanic.dta if you haven't already
	\item Introduce new Stata commands and practice using them in the Command window
	\item Work in pairs to finish class8.do that you started for the pre class exercise
\end{itemize}
\end{frame}

\begin{frame}
\frametitle{String vs numeric variables}
Last class we learned how to generate a numeric variable. Let's try generating a string variable:
\begin{itemize}
	\item \texttt{gen crew = "no"}
	\item \texttt{replace crew = "yes" if class == 0} \pause \bigskip
\end{itemize}
Text strings must \textit{always} be bounded by quotes. \\ \pause \bigskip
Use \texttt{desc} to check the storage type of each variable.
\end{frame}

\begin{frame}
\frametitle{Some notes on naming variables}
\begin{itemize} \itemsep1em
\item No spaces allowed, case sensitive
\item Up to 32 characters
\item Rules of thumb
\begin{itemize} \pause
\item Keep it short, e.g. \textit{educ, income}
\item Make it informative: e.g. \textit{female} instead of \textit{gender}
\item Maintain consistency, e.g. \textit{ln\_income}, \textit{ln\_wage}, \textit{ln\_tax}
\end{itemize}
\end{itemize}
\end{frame}

\begin{frame}
\frametitle{Command: \texttt{egen}}
\texttt{egen} is like \texttt{gen}, but it's used for assigning values to variables with functions that work across all observations. Try:
        \begin{center}
            \texttt{\alert{gen} mean\_survive = mean(survive)} \\
            \pause
            \bigskip
        \end{center}
 The above fails to work. You must use \texttt{egen}:
        \begin{center}
            \texttt{\alert{egen} mean\_survive = mean(survive)}
        \end{center}
\end{frame}

\begin{frame}
\frametitle{Generating variables by group}
Suppose we want to calculate the mean survival rate by class. We can accomplish this using \texttt{sort}, \texttt{by}, and \texttt{egen}:
\begin{itemize}
	\item \texttt{sort class id}
	\item \texttt{by class: egen mean\_survive\_class = mean(survive)}
\end{itemize} \pause \bigskip
\texttt{sort} orders the data from smallest to largest numerically or alphabetically (if string). If multiple variables are listed, the data are sorted lexicographically.  \\ \pause \bigskip
\texttt{by class} tells the program to take the mean within each class separately. The data must be sorted to apply commands within designated groups. 
\end{frame}

\begin{frame}
\frametitle{Commands: \texttt{if}, \texttt{in}, \& \texttt{\underline{l}ist}}
\begin{itemize}
	\item \texttt{if} restricts the command to certain criteria, \textit{for each observation}, similar to IF in Excel \pause
	\item \texttt{in} restricts the observations to which a function applies. \pause
	\item \texttt{list} displays observations for specified variables or all variables if unspecified.  
\end{itemize} \medskip \pause
Try it:
\begin{itemize} 
	\item[] \texttt{list in 1/200 if male == 0}
	\item[] \texttt{list id survive in 1/100 if male == 1}
\end{itemize} 
\end{frame}

\begin{frame}
\frametitle{Commands: \texttt{keep} \& \texttt{drop}}
Both of these commands are used to keep or drop (delete) observations or variables. Try the following:
\begin{itemize}
	\item \texttt{keep in 1/2000}
	\item \texttt{drop in 1001/2000}
	\item \texttt{keep id adult survive}
	\item \texttt{drop adult}
\end{itemize} \pause \bigskip
Remember: these changes are made to the data in working memory. The original dataset (titanic.dta) remains unchanged unless you save it. \\ \pause \bigskip
Go ahead and restore the data with: \texttt{use titanic, clear}
\end{frame}

\begin{frame}
\frametitle{Logic in Stata}
\begin{itemize}
    \item \texttt{\&} means `AND' ; \texttt{$\vert$} means `OR' ; \texttt{!} means `NOT'
	\item Use parenthesis when necessary. In Stata, \texttt{\&} takes precedence over \texttt{$\vert$}
\end{itemize} \pause
\begin{center}
	\texttt{a $\vert$ b \& c} is \textbf{not} the same as \texttt{(a $\vert$ b) \& c} \\ \bigskip \pause
	\texttt{a $\vert$ b \& c} \textbf{is} the same as \texttt{a $\vert$ (b \& c)}
\end{center} \pause \bigskip
Remember, use double equals signs to perform a logical test, \\ e.g. \texttt{list if id == 100} \\ \pause \bigskip
Use single equals signs to assign values, e.g. \texttt{gen var = 100}
\end{frame}

%\begin{frame}
%    \frametitle{Missing entries are a pain}
%    \begin{itemize}
%        \item There are 27 missing values in Stata
%        \item These are all larger than any possible number in Stata
%        \item The `smallest' missing value is $. < .a < \cdots < .z$
%        \item Two ways to deal with missing:
%            \begin{itemize}
%                \item Somehow logically deal with them: list if price $<$ .
%                \item Or: list if !missing(price)
%            \end{itemize}
%    \end{itemize}
%\end{frame}

\begin{frame}
\frametitle{Time to work on class8.do}
Here are the commands/operators we covered today:
		\begin{itemize}
			\item \texttt{egen}
			\item \texttt{if}
			\item \texttt{in}
			\item \texttt{list}
			\item \texttt{keep}
			\item \texttt{drop}
			\item \texttt{\&, $\vert$, !}
			\item \texttt{=} vs \texttt{==}
		\end{itemize}
\end{frame}
\end{document}









