\documentclass[ignorenonframetext,]{beamer}
\setbeamertemplate{caption}[numbered]
\setbeamertemplate{caption label separator}{: }
\setbeamercolor{caption name}{fg=normal text.fg}
\beamertemplatenavigationsymbolsempty
\usepackage{lmodern}
\usepackage{amssymb,amsmath}
\usepackage{ifxetex,ifluatex}
\usepackage{fixltx2e} % provides \textsubscript

\usetheme{CambridgeUS}
\definecolor{UBCblue}{rgb}{0.04706, 0.13725, 0.26667}
\definecolor{UBCgray}{rgb}{0.3686, 0.5255, 0.6235}
\colorlet{verylightgray}{gray!10}
\setbeamercolor{palette primary}{bg=UBCblue,fg=white}
\setbeamercolor{palette secondary}{bg=darkgray,fg=white}
\setbeamercolor{palette tertiary}{bg=UBCblue,fg=white}
\setbeamercolor{palette quaternary}{bg=UBCblue,fg=white}
\setbeamercolor{structure}{fg=UBCblue} % itemize, enumerate, etc
\setbeamercolor{section in toc}{fg=UBCblue} % TOC sections
\setbeamercolor{subsection in head/foot}{bg=darkgray,fg=white}
\setbeamercolor{frametitle}{fg=UBCblue}
\setbeamercolor{title}{fg=UBCblue, bg=verylightgray}
\setbeamertemplate{itemize items}{\color{UBCblue}$\blacktriangleright$}


\ifnum 0\ifxetex 1\fi\ifluatex 1\fi=0 % if pdftex
\usepackage[T1]{fontenc}
\usepackage[utf8]{inputenc}
\else % if luatex or xelatex
\ifxetex
\usepackage{mathspec}
\else
\usepackage{fontspec}
\fi
\defaultfontfeatures{Ligatures=TeX,Scale=MatchLowercase}
\fi
% use upquote if available, for straight quotes in verbatim environments
\IfFileExists{upquote.sty}{\usepackage{upquote}}{}
% use microtype if available
\IfFileExists{microtype.sty}{%
	\usepackage{microtype}
	\UseMicrotypeSet[protrusion]{basicmath} % disable protrusion for tt fonts
}{}
\newif\ifbibliography
\hypersetup{
	pdfauthor={UCSD},
	pdfborder={0 0 0},
	breaklinks=true}
\urlstyle{same}  % don't use monospace font for urls
\usepackage{color}
\usepackage{fancyvrb}
\newcommand{\VerbBar}{|}
\newcommand{\VERB}{\Verb[commandchars=\\\{\}]}
\DefineVerbatimEnvironment{Highlighting}{Verbatim}{commandchars=\\\{\}}
% Add ',fontsize=\small' for more characters per line
\usepackage{framed}
\definecolor{shadecolor}{RGB}{248,248,248}
\newenvironment{Shaded}{\begin{snugshade}}{\end{snugshade}}
\newcommand{\KeywordTok}[1]{\textcolor[rgb]{0.13,0.29,0.53}{\textbf{#1}}}
\newcommand{\DataTypeTok}[1]{\textcolor[rgb]{0.13,0.29,0.53}{#1}}
\newcommand{\DecValTok}[1]{\textcolor[rgb]{0.00,0.00,0.81}{#1}}
\newcommand{\BaseNTok}[1]{\textcolor[rgb]{0.00,0.00,0.81}{#1}}
\newcommand{\FloatTok}[1]{\textcolor[rgb]{0.00,0.00,0.81}{#1}}
\newcommand{\ConstantTok}[1]{\textcolor[rgb]{0.00,0.00,0.00}{#1}}
\newcommand{\CharTok}[1]{\textcolor[rgb]{0.31,0.60,0.02}{#1}}
\newcommand{\SpecialCharTok}[1]{\textcolor[rgb]{0.00,0.00,0.00}{#1}}
\newcommand{\StringTok}[1]{\textcolor[rgb]{0.31,0.60,0.02}{#1}}
\newcommand{\VerbatimStringTok}[1]{\textcolor[rgb]{0.31,0.60,0.02}{#1}}
\newcommand{\SpecialStringTok}[1]{\textcolor[rgb]{0.31,0.60,0.02}{#1}}
\newcommand{\ImportTok}[1]{#1}
\newcommand{\CommentTok}[1]{\textcolor[rgb]{0.56,0.35,0.01}{\textit{#1}}}
\newcommand{\DocumentationTok}[1]{\textcolor[rgb]{0.56,0.35,0.01}{\textbf{\textit{#1}}}}
\newcommand{\AnnotationTok}[1]{\textcolor[rgb]{0.56,0.35,0.01}{\textbf{\textit{#1}}}}
\newcommand{\CommentVarTok}[1]{\textcolor[rgb]{0.56,0.35,0.01}{\textbf{\textit{#1}}}}
\newcommand{\OtherTok}[1]{\textcolor[rgb]{0.56,0.35,0.01}{#1}}
\newcommand{\FunctionTok}[1]{\textcolor[rgb]{0.00,0.00,0.00}{#1}}
\newcommand{\VariableTok}[1]{\textcolor[rgb]{0.00,0.00,0.00}{#1}}
\newcommand{\ControlFlowTok}[1]{\textcolor[rgb]{0.13,0.29,0.53}{\textbf{#1}}}
\newcommand{\OperatorTok}[1]{\textcolor[rgb]{0.81,0.36,0.00}{\textbf{#1}}}
\newcommand{\BuiltInTok}[1]{#1}
\newcommand{\ExtensionTok}[1]{#1}
\newcommand{\PreprocessorTok}[1]{\textcolor[rgb]{0.56,0.35,0.01}{\textit{#1}}}
\newcommand{\AttributeTok}[1]{\textcolor[rgb]{0.77,0.63,0.00}{#1}}
\newcommand{\RegionMarkerTok}[1]{#1}
\newcommand{\InformationTok}[1]{\textcolor[rgb]{0.56,0.35,0.01}{\textbf{\textit{#1}}}}
\newcommand{\WarningTok}[1]{\textcolor[rgb]{0.56,0.35,0.01}{\textbf{\textit{#1}}}}
\newcommand{\AlertTok}[1]{\textcolor[rgb]{0.94,0.16,0.16}{#1}}
\newcommand{\ErrorTok}[1]{\textcolor[rgb]{0.64,0.00,0.00}{\textbf{#1}}}
\newcommand{\NormalTok}[1]{#1}
\usepackage{graphicx,grffile}
\makeatletter
\def\maxwidth{\ifdim\Gin@nat@width>\linewidth\linewidth\else\Gin@nat@width\fi}
\def\maxheight{\ifdim\Gin@nat@height>\textheight0.8\textheight\else\Gin@nat@height\fi}
\makeatother
% Scale images if necessary, so that they will not overflow the page
% margins by default, and it is still possible to overwrite the defaults
% using explicit options in \includegraphics[width, height, ...]{}
\setkeys{Gin}{width=\maxwidth,height=\maxheight,keepaspectratio}

% Prevent slide breaks in the middle of a paragraph:
\widowpenalties 1 10000
\raggedbottom

\AtBeginPart{
	\let\insertpartnumber\relax
	\let\partname\relax
	\frame{\partpage}
}
\AtBeginSection{
	\ifbibliography
	\else
	\let\insertsectionnumber\relax
	\let\sectionname\relax
	\frame{\sectionpage}
	\fi
}
\AtBeginSubsection{
	\let\insertsubsectionnumber\relax
	\let\subsectionname\relax
	\frame{\subsectionpage}
}

\setlength{\parindent}{0pt}
\setlength{\parskip}{6pt plus 2pt minus 1pt}
\setlength{\emergencystretch}{3em}  % prevent overfull lines
\providecommand{\tightlist}{%
	\setlength{\itemsep}{0pt}\setlength{\parskip}{0pt}}
\setcounter{secnumdepth}{0}


\title[Class 28]{Introduction to Social Data Analytics\\
	Class 28}
\author[Kaushik]{Arushi Kaushik}
\institute[UCSD]{arkaushi@ucsd.edu}
%\date[]{}

\begin{document}
\frame{\titlepage}

\begin{frame}[fragile]{Today: \texttt{dplyr} package}

By the end of today's lecture, you should be able to:

\begin{itemize}
\tightlist
\item
  Download R packages from CRAN (Comprehensive R Archive Network) and load them into your session
\item
  Use the \texttt{dplyr} package to accomplish basic data wrangling
  including:

  \begin{itemize}
  \tightlist
  \item
    Adding and deleting variables and observations
  \item
    Sorting, merging and reshaping data
  \item
    Collapsing a table to a coarser unit of analysis
  \end{itemize}
\end{itemize}

Open class28.R if you haven't already.

\end{frame}

\begin{frame}{What is a package?}

\begin{itemize}
\tightlist
\item
  A package is a collection of R functions, data, and code that expand
  the capabilities of base R.
\item
  Anyone can create a package - there's even a package to help with
  creating packages.
\end{itemize}

There are more than 14,000 publicly available packages that you can
download from the Comprehensive R Archive Network (CRAN).

\end{frame}

\begin{frame}[fragile]{Using a package requires two steps.}

\begin{itemize}
\tightlist
\item
  First, install the package using:
\end{itemize}

\begin{Shaded}
\begin{Highlighting}[]
\KeywordTok{install.packages}\NormalTok{(dplyr)}
\end{Highlighting}
\end{Shaded}

You only need to download a package once. It'll remain on your computer
until you uninstall R.

\begin{itemize}
\tightlist
\item
  Second, load the package using:
\end{itemize}

\begin{Shaded}
\begin{Highlighting}[]
\KeywordTok{library}\NormalTok{(dplyr)}
\end{Highlighting}
\end{Shaded}

You must load a package once per session.

\end{frame}

\begin{frame}[fragile]{Let's create variables with \texttt{mutate()}}

Create a variable \texttt{region\$california\_\ =\ TRUE} if
\texttt{region\$region\ ==\ "California"}, \texttt{FALSE} otherwise,
using:

\begin{enumerate}
\def\labelenumi{\arabic{enumi}.}
\tightlist
\item[1.]
  the old fashioned way
\item[2.]
  the dpylr function `mutate()'
\end{enumerate}

\begin{Shaded}
\begin{Highlighting}[]
\CommentTok{# a)}
\NormalTok{region}\OperatorTok{$}\NormalTok{california_a <-}\StringTok{ }\NormalTok{region}\OperatorTok{$}\NormalTok{region }\OperatorTok{==}\StringTok{ "California"}
\end{Highlighting}
\end{Shaded}

\end{frame}

\begin{frame}[fragile]{Let's create variables with \texttt{mutate()}}

\begin{Shaded}
\begin{Highlighting}[]
\CommentTok{# a)}
\NormalTok{region <-}\StringTok{ }\NormalTok{region }\OperatorTok\StringTok{ }
\StringTok{  }\KeywordTok{mutate}\NormalTok{(}\DataTypeTok{california_b =}\NormalTok{ region }\OperatorTok{==}\StringTok{ "California"}\NormalTok{)}

\CommentTok{# compare the two:}
\KeywordTok{table}\NormalTok{(region}\OperatorTok{$}\NormalTok{california_a, region}\OperatorTok{$}\NormalTok{california_b)}
\end{Highlighting}
\end{Shaded}

\begin{verbatim}
##        
##         FALSE TRUE
##   FALSE   292    0
##   TRUE      0   28
\end{verbatim}

\end{frame}

\begin{frame}[fragile]{Your turn:}

Add the variable \texttt{salary\$high} =

\begin{itemize}
\tightlist
\item
  1 if \texttt{salary\_start\_median\ \textgreater{}\ 50000},
\item
  0 otherwise
\end{itemize}

using \texttt{mutate()}.

\end{frame}

\begin{frame}[fragile]{Solution:}

\begin{Shaded}
\begin{Highlighting}[]
\NormalTok{salary <-}\StringTok{ }\NormalTok{salary }\OperatorTok\StringTok{ }
\StringTok{  }\KeywordTok{mutate}\NormalTok{(}\DataTypeTok{high =} \KeywordTok{as.numeric}\NormalTok{(salary_start_median }\OperatorTok{>}\StringTok{ }\DecValTok{50000}\NormalTok{))}

\NormalTok{salary[}\DecValTok{6}\OperatorTok{:}\DecValTok{10}\NormalTok{, }\KeywordTok{c}\NormalTok{(}\StringTok{"salary_start_median"}\NormalTok{, }\StringTok{"high"}\NormalTok{)]}
\end{Highlighting}
\end{Shaded}

\begin{verbatim}
## # A tibble: 5 x 2
##   salary_start_median  high
##                 <dbl> <dbl>
## 1               57200     1
## 2               52600     1
## 3               51100     1
## 4               48600     0
## 5               54800     1
\end{verbatim}

\end{frame}

\begin{frame}[fragile]{Multiple functions at once with piping}

One cool aspect of piping is the ability to apply multiple functions at
once.

Below we apply two functons to the data frame `salary':

\begin{enumerate}
\def\labelenumi{\arabic{enumi}.}
\tightlist
\item
  We add a variable salary\_thousands = salary\_start\_median / 1000
\item
  We rename the variable `salary' to `california'
\end{enumerate}

\begin{Shaded}
\begin{Highlighting}[]
\NormalTok{salary <-}\StringTok{ }\NormalTok{salary }\OperatorTok
\StringTok{  }\KeywordTok{mutate}\NormalTok{(}\DataTypeTok{salary_thousands =}\NormalTok{ salary_start_median}\OperatorTok{/}\DecValTok{1000}\NormalTok{) }\OperatorTok
\StringTok{  }\KeywordTok{rename}\NormalTok{(}\DataTypeTok{salary_90 =}\NormalTok{ salary_midcareer_90th)}
\end{Highlighting}
\end{Shaded}

It can be helpful to read the \texttt{\%\textgreater{}\%}s as ``and
then''.

\end{frame}

\begin{frame}[fragile]{Your turn:}

Rename the variable \texttt{salary\$high} to
\texttt{salary\$salary\_high} using rename().

\end{frame}

\begin{frame}[fragile]{Solution}

\begin{Shaded}
\begin{Highlighting}[]
\NormalTok{salary <-}\StringTok{ }\NormalTok{salary }\OperatorTok\StringTok{ }\KeywordTok{rename}\NormalTok{(}\DataTypeTok{salary_high =}\NormalTok{ high)}
\end{Highlighting}
\end{Shaded}

\end{frame}

\begin{frame}[fragile]{Keeping certain \textit{observations}}

Sometimes we want to subset a data frame that only includes certain
observations.

\begin{itemize}
\tightlist
\item
  filter() keeps observations that meet logical criteria
\item
  distinct() removes observations with duplicate values
\end{itemize}

\begin{Shaded}
\begin{Highlighting}[]
\NormalTok{state_schools <-}\StringTok{ }\NormalTok{type }\OperatorTok\StringTok{ }\KeywordTok{filter}\NormalTok{(type }\OperatorTok{==}\StringTok{ "State"}\NormalTok{)}

\KeywordTok{head}\NormalTok{(state_schools, }\DecValTok{3}\NormalTok{)}
\end{Highlighting}
\end{Shaded}

\begin{verbatim}
## # A tibble: 3 x 2
##   name                            type 
##   <chr>                           <chr>
## 1 Appalachian State University    State
## 2 Arkansas State University (ASU) State
## 3 Auburn University               State
\end{verbatim}

\end{frame}

\begin{frame}[fragile]{Keeping certain \textit{observations}}

\begin{Shaded}
\begin{Highlighting}[]
\NormalTok{unique_regions <-}\StringTok{ }\NormalTok{region }\OperatorTok\StringTok{ }\KeywordTok{distinct}\NormalTok{(region)}

\KeywordTok{head}\NormalTok{(unique_regions)}
\end{Highlighting}
\end{Shaded}

\begin{verbatim}
## # A tibble: 5 x 1
##   region      
##   <chr>       
## 1 California  
## 2 Western     
## 3 Midwestern  
## 4 Southern    
## 5 Northeastern
\end{verbatim}

\end{frame}

\begin{frame}[fragile]{Your turn:}

\begin{enumerate}
\def\labelenumi{\arabic{enumi}.}
\item[1.]
  Create a table called `rich.grads' that contains all observations
  within `salary' where \texttt{salary\_90\ \textgreater{}\ 200000}. Use
  \texttt{filter()}.
\item[2.]
  Create a table called `unique\_types' that contains one observation
  per `type' in the data frame `type'. Use \texttt{distinct()}.
\end{enumerate}

\end{frame}

\begin{frame}[fragile]{Solution:}

\begin{Shaded}
\begin{Highlighting}[]
\NormalTok{rich.grads <-}\StringTok{ }\NormalTok{salary }\OperatorTok\StringTok{ }\KeywordTok{filter}\NormalTok{(salary_}\DecValTok{90} \OperatorTok{>}\StringTok{ }\DecValTok{200000}\NormalTok{)}

\NormalTok{rich.grads[}\DecValTok{1}\OperatorTok{:}\DecValTok{5}\NormalTok{, }\KeywordTok{c}\NormalTok{(}\StringTok{"name"}\NormalTok{, }\StringTok{"salary_90"}\NormalTok{)]}
\end{Highlighting}
\end{Shaded}

\begin{verbatim}
## # A tibble: 5 x 2
##   name                                    salary_90
##   <chr>                                       <dbl>
## 1 Stanford University                        257000
## 2 University of California, Berkeley         201000
## 3 University of Southern California (USC)    201000
## 4 University of California, Davis            202000
## 5 Colorado School of Mines                   201000
\end{verbatim}

\end{frame}

\begin{frame}[fragile]{Solution:}

\begin{Shaded}
\begin{Highlighting}[]
\NormalTok{unique_types <-}\StringTok{ }\NormalTok{type }\OperatorTok\StringTok{ }\KeywordTok{distinct}\NormalTok{(type)}

\NormalTok{unique_types}
\end{Highlighting}
\end{Shaded}

\begin{verbatim}
## # A tibble: 5 x 1
##   type        
##   <chr>       
## 1 Liberal Arts
## 2 State       
## 3 Party       
## 4 Ivy League  
## 5 Engineering
\end{verbatim}

\end{frame}

\begin{frame}[fragile]{Keeping certain \textit{variables}}

\begin{itemize}
\tightlist
\item
  select() keeps only the variables listed
\item
  transmute() keeps only variables listed and allows creation of new
  variables
\end{itemize}

\begin{Shaded}
\begin{Highlighting}[]
\NormalTok{salary_median <-}\StringTok{ }\NormalTok{salary }\OperatorTok\StringTok{ }
\StringTok{  }\KeywordTok{select}\NormalTok{(name, salary_start_median, salary_midcareer_median)}

\KeywordTok{head}\NormalTok{(salary_median, }\DecValTok{3}\NormalTok{)}
\end{Highlighting}
\end{Shaded}

\begin{verbatim}
## # A tibble: 3 x 3
##   name                             salary_start_medi~ salary_midcareer_med~
##   <chr>                                         <dbl>                 <dbl>
## 1 Stanford University                           70400                129000
## 2 California Institute of Technol~              75500                123000
## 3 Harvey Mudd College                           71800                122000
\end{verbatim}

\end{frame}

\begin{frame}[fragile]{Keeping certain variables}

\begin{Shaded}
\begin{Highlighting}[]
\NormalTok{salary_median_thous <-}\StringTok{ }\NormalTok{salary }\OperatorTok\StringTok{ }
\StringTok{  }\KeywordTok{transmute}\NormalTok{(}\DataTypeTok{name =}\NormalTok{ name, }
      \DataTypeTok{salary_start_thous =}\NormalTok{ salary_start_median}\OperatorTok{/}\DecValTok{1000}\NormalTok{, }
      \DataTypeTok{salary_midcareer_thous =}\NormalTok{ salary_midcareer_median}\OperatorTok{/}\DecValTok{1000}\NormalTok{)}

\KeywordTok{head}\NormalTok{(salary_median_thous, }\DecValTok{3}\NormalTok{)}
\end{Highlighting}
\end{Shaded}

\begin{verbatim}
## # A tibble: 3 x 3
##   name                              salary_start_thous salary_midcareer_th~
##   <chr>                                          <dbl>                <dbl>
## 1 Stanford University                             70.4                  129
## 2 California Institute of Technolo~               75.5                  123
## 3 Harvey Mudd College                             71.8                  122
\end{verbatim}

\end{frame}

\begin{frame}[fragile]{Your turn:}

Overwrite the table `rich.grads' to only contain the variables:

\begin{itemize}
\tightlist
\item
  \texttt{name}\\
\item
  \texttt{salary\_90\_thous\ =\ salary\_/1000}.
\end{itemize}

\end{frame}

\begin{frame}[fragile]{Solution:}

\begin{Shaded}
\begin{Highlighting}[]
\NormalTok{rich.grads <-}\StringTok{ }\NormalTok{rich.grads }\OperatorTok\StringTok{ }
\StringTok{  }\KeywordTok{transmute}\NormalTok{(}\DataTypeTok{name =}\NormalTok{ name, }\DataTypeTok{salary_90_thous =}\NormalTok{ salary_}\DecValTok{90}\OperatorTok{/}\DecValTok{1000}\NormalTok{)}

\KeywordTok{head}\NormalTok{(rich.grads)}
\end{Highlighting}
\end{Shaded}

\begin{verbatim}
## # A tibble: 6 x 2
##   name                                    salary_90_thous
##   <chr>                                             <dbl>
## 1 Stanford University                                 257
## 2 University of California, Berkeley                  201
## 3 University of Southern California (USC)             201
## 4 University of California, Davis                     202
## 5 Colorado School of Mines                            201
## 6 University of Notre Dame                            235
\end{verbatim}

\end{frame}

\begin{frame}[fragile]{Sorting data in ascending order}

We can sort data using \texttt{arrange()}.

\begin{Shaded}
\begin{Highlighting}[]
\NormalTok{salary <-}\StringTok{ }\NormalTok{salary }\OperatorTok\StringTok{ }\KeywordTok{arrange}\NormalTok{(salary_start_median)}

\KeywordTok{head}\NormalTok{(salary, }\DecValTok{3}\NormalTok{) }\CommentTok{# lowest to highest}
\end{Highlighting}
\end{Shaded}

\begin{verbatim}
## # A tibble: 3 x 9
##   name  salary_start_me~ salary_midcaree~ salary_midcaree~ salary_midcaree~
##   <chr>            <dbl>            <dbl>            <dbl>            <dbl>
## 1 Lee ~            34500            53900               NA            44500
## 2 Virg~            34600            54900               NA            37100
## 3 More~            34800            60600            34300            46500
## # ... with 4 more variables: salary_midcareer_75th <dbl>, salary_90 <dbl>,
## #   salary_high <dbl>, salary_thousands <dbl>
\end{verbatim}

\end{frame}

\begin{frame}[fragile]{Sorting data in descending order}

\begin{Shaded}
\begin{Highlighting}[]
\NormalTok{salary <-}\StringTok{ }\NormalTok{salary }\OperatorTok\StringTok{ }\KeywordTok{arrange}\NormalTok{(}\KeywordTok{desc}\NormalTok{(salary_start_median))}

\KeywordTok{head}\NormalTok{(salary, }\DecValTok{3}\NormalTok{) }\CommentTok{# highest to lowest}
\end{Highlighting}
\end{Shaded}

\begin{verbatim}
## # A tibble: 3 x 9
##   name  salary_start_me~ salary_midcaree~ salary_midcaree~ salary_midcaree~
##   <chr>            <dbl>            <dbl>            <dbl>            <dbl>
## 1 Cali~            75500           123000               NA           104000
## 2 Mass~            72200           126000            76800            99200
## 3 Harv~            71800           122000               NA            96000
## # ... with 4 more variables: salary_midcareer_75th <dbl>, salary_90 <dbl>,
## #   salary_high <dbl>, salary_thousands <dbl>
\end{verbatim}

\end{frame}

\begin{frame}[fragile]{Your turn:}

Sort \texttt{rich.grads} to go from highest 90th percentile salary to
lowest.

\end{frame}

\begin{frame}[fragile]{Solution:}

\begin{Shaded}
\begin{Highlighting}[]
\NormalTok{rich.grads <-}\StringTok{ }\NormalTok{rich.grads }\OperatorTok\StringTok{ }
\StringTok{  }\KeywordTok{arrange}\NormalTok{(}\KeywordTok{desc}\NormalTok{(salary_90_thous))}

\KeywordTok{head}\NormalTok{(rich.grads) }\CommentTok{# highest to lowest}
\end{Highlighting}
\end{Shaded}

\begin{verbatim}
## # A tibble: 6 x 2
##   name                       salary_90_thous
##   <chr>                                <dbl>
## 1 Yale University                        326
## 2 Dartmouth College                      321
## 3 Harvard University                     288
## 4 University of Pennsylvania             282
## 5 Colgate University                     265
## 6 Princeton University                   261
\end{verbatim}

\end{frame}

\begin{frame}[fragile]{Merging data}

We can join data tables using a common identifier, in our case `name'
(note the \emph{left} join).

\begin{Shaded}
\begin{Highlighting}[]
\NormalTok{df <-}\StringTok{ }\NormalTok{salary }\OperatorTok\StringTok{ }\KeywordTok{left_join}\NormalTok{(region, }\DataTypeTok{by =} \StringTok{"name"}\NormalTok{)}
\end{Highlighting}
\end{Shaded}

After that, we will left join `type':

\begin{Shaded}
\begin{Highlighting}[]
\NormalTok{df <-}\StringTok{ }\NormalTok{df }\OperatorTok\StringTok{ }\KeywordTok{left_join}\NormalTok{(type, }\DataTypeTok{by =} \StringTok{"name"}\NormalTok{)}
\KeywordTok{names}\NormalTok{(df)}
\end{Highlighting}
\end{Shaded}

\begin{verbatim}
## [1] "name"                    "salary_start_median"    
## [3] "salary_midcareer_median" "salary_midcareer_10th"  
## [5] "salary_midcareer_25th"   "salary_midcareer_75th"  
## [7] "salary_midcareer_90th"   "region"                 
## [9] "type"
\end{verbatim}

\end{frame}

\begin{frame}[fragile]{Two merges in one line of code:}

\begin{Shaded}
\begin{Highlighting}[]
\NormalTok{df <-}\StringTok{ }\NormalTok{salary }\OperatorTok\StringTok{ }
\StringTok{  }\KeywordTok{left_join}\NormalTok{(region, }\DataTypeTok{by =} \StringTok{"name"}\NormalTok{) }\OperatorTok\StringTok{ }
\StringTok{  }\KeywordTok{left_join}\NormalTok{(type, }\DataTypeTok{by =} \StringTok{"name"}\NormalTok{)}

\KeywordTok{names}\NormalTok{(df)}
\end{Highlighting}
\end{Shaded}

\begin{verbatim}
## [1] "name"                    "salary_start_median"    
## [3] "salary_midcareer_median" "salary_midcareer_10th"  
## [5] "salary_midcareer_25th"   "salary_midcareer_75th"  
## [7] "salary_midcareer_90th"   "region"                 
## [9] "type"
\end{verbatim}

\end{frame}

\begin{frame}[fragile]{Group-wise operations}

We can apply functions after using the group\_by() function. What's the
new unit of analysis in this example?

\begin{Shaded}
\begin{Highlighting}[]
\NormalTok{df_region <-}\StringTok{ }\NormalTok{df }\OperatorTok\StringTok{ }
\StringTok{  }\KeywordTok{group_by}\NormalTok{(region) }\OperatorTok\StringTok{ }
\StringTok{  }\KeywordTok{summarise}\NormalTok{(}\DataTypeTok{mean_start =} \KeywordTok{mean}\NormalTok{(salary_start_median), }
            \DataTypeTok{mean_mid =} \KeywordTok{mean}\NormalTok{(salary_midcareer_median))}

\KeywordTok{head}\NormalTok{(df_region, }\DecValTok{3}\NormalTok{)}
\end{Highlighting}
\end{Shaded}

\begin{verbatim}
## # A tibble: 3 x 3
##   region       mean_start mean_mid
##   <chr>             <dbl>    <dbl>
## 1 California       51032.   93132.
## 2 Midwestern       44225.   78180.
## 3 Northeastern     48496    91352
\end{verbatim}

\end{frame}

\begin{frame}[fragile]{Your turn:}

Create \texttt{df\_type} that is a data frame at the type-level and
contains the average starting and mid career salaries by univeristy
type.

\end{frame}

\begin{frame}[fragile]{Solution:}

\begin{Shaded}
\begin{Highlighting}[]
\NormalTok{df_type <-}\StringTok{ }\NormalTok{df }\OperatorTok\StringTok{ }
\StringTok{  }\KeywordTok{group_by}\NormalTok{(type) }\OperatorTok\StringTok{ }
\StringTok{  }\KeywordTok{summarise}\NormalTok{(}\DataTypeTok{mean_start =} \KeywordTok{mean}\NormalTok{(salary_start_median), }
            \DataTypeTok{mean_mid =} \KeywordTok{mean}\NormalTok{(salary_midcareer_median))}

\KeywordTok{head}\NormalTok{(df_type, }\DecValTok{3}\NormalTok{)}
\end{Highlighting}
\end{Shaded}

\begin{verbatim}
## # A tibble: 3 x 3
##   type        mean_start mean_mid
##   <chr>            <dbl>    <dbl>
## 1 <NA>            46885.   84106.
## 2 Engineering     59411.  105128.
## 3 Ivy League      60475   120125
\end{verbatim}

\end{frame}

\begin{frame}[fragile]{Summary}

class28.R contains an example of reshaping in \texttt{R} using another
package, \texttt{tidyr}. Also check out the data-wrangling cheat sheet
on TritonEd.

Here are the commands/operators we covered today:

\begin{columns}
	\begin{column}{0.5\textwidth}
		\begin{itemize}
			\item \texttt{install.packages}
			\item \texttt{library}
			\item \texttt{mutate}
			\item \texttt{rename}
			\item \texttt{filter}
			\item \texttt{distinct}
			\item \texttt{select}
			\item \texttt{transmute}
		\end{itemize}
	\end{column}
	\begin{column}{0.5\textwidth}
		\begin{itemize}
			\item \texttt{arrange}
			\item \texttt{desc}
			\item \texttt{left\_join}
			\item \texttt{group\_by}
			\item \texttt{summarise}
			\item \texttt{everything}
			\item \texttt{gather}
			\item \texttt{spread}
		\end{itemize}
	\end{column}
\end{columns}

\end{frame}

\end{document}
