%\documentclass[11pt]{beamer}
%\mode<presentation>
\documentclass[handout,compress]{beamer}
\usepackage{beamerthemedefault}
\usepackage{graphicx}
\usepackage{hyperref}
\usepackage{subfigure}
\usepackage{color}
\usepackage{multicol}
\usepackage{bm} 
\usepackage{tikz}
\usepackage{listliketab}

\usepackage{framed}
\definecolor{shadecolor}{RGB}{248,248,248}
\newenvironment{Shaded}{\begin{snugshade}}{\end{snugshade}}
\newcommand{\KeywordTok}[1]{\textcolor[rgb]{0.13,0.29,0.53}{\textbf{#1}}}
\newcommand{\DataTypeTok}[1]{\textcolor[rgb]{0.13,0.29,0.53}{#1}}
\newcommand{\DecValTok}[1]{\textcolor[rgb]{0.00,0.00,0.81}{#1}}
\newcommand{\BaseNTok}[1]{\textcolor[rgb]{0.00,0.00,0.81}{#1}}
\newcommand{\FloatTok}[1]{\textcolor[rgb]{0.00,0.00,0.81}{#1}}
\newcommand{\ConstantTok}[1]{\textcolor[rgb]{0.00,0.00,0.00}{#1}}
\newcommand{\CharTok}[1]{\textcolor[rgb]{0.31,0.60,0.02}{#1}}
\newcommand{\SpecialCharTok}[1]{\textcolor[rgb]{0.00,0.00,0.00}{#1}}
\newcommand{\StringTok}[1]{\textcolor[rgb]{0.31,0.60,0.02}{#1}}
\newcommand{\VerbatimStringTok}[1]{\textcolor[rgb]{0.31,0.60,0.02}{#1}}
\newcommand{\SpecialStringTok}[1]{\textcolor[rgb]{0.31,0.60,0.02}{#1}}
\newcommand{\ImportTok}[1]{#1}
\newcommand{\CommentTok}[1]{\textcolor[rgb]{0.56,0.35,0.01}{\textit{#1}}}
\newcommand{\DocumentationTok}[1]{\textcolor[rgb]{0.56,0.35,0.01}{\textbf{\textit{#1}}}}
\newcommand{\AnnotationTok}[1]{\textcolor[rgb]{0.56,0.35,0.01}{\textbf{\textit{#1}}}}
\newcommand{\CommentVarTok}[1]{\textcolor[rgb]{0.56,0.35,0.01}{\textbf{\textit{#1}}}}
\newcommand{\OtherTok}[1]{\textcolor[rgb]{0.56,0.35,0.01}{#1}}
\newcommand{\FunctionTok}[1]{\textcolor[rgb]{0.00,0.00,0.00}{#1}}
\newcommand{\VariableTok}[1]{\textcolor[rgb]{0.00,0.00,0.00}{#1}}
\newcommand{\ControlFlowTok}[1]{\textcolor[rgb]{0.13,0.29,0.53}{\textbf{#1}}}
\newcommand{\OperatorTok}[1]{\textcolor[rgb]{0.81,0.36,0.00}{\textbf{#1}}}
\newcommand{\BuiltInTok}[1]{#1}
\newcommand{\ExtensionTok}[1]{#1}
\newcommand{\PreprocessorTok}[1]{\textcolor[rgb]{0.56,0.35,0.01}{\textit{#1}}}
\newcommand{\AttributeTok}[1]{\textcolor[rgb]{0.77,0.63,0.00}{#1}}
\newcommand{\RegionMarkerTok}[1]{#1}
\newcommand{\InformationTok}[1]{\textcolor[rgb]{0.56,0.35,0.01}{\textbf{\textit{#1}}}}
\newcommand{\WarningTok}[1]{\textcolor[rgb]{0.56,0.35,0.01}{\textbf{\textit{#1}}}}
\newcommand{\AlertTok}[1]{\textcolor[rgb]{0.94,0.16,0.16}{#1}}
\newcommand{\ErrorTok}[1]{\textcolor[rgb]{0.64,0.00,0.00}{\textbf{#1}}}
\newcommand{\NormalTok}[1]{#1}

\usetheme{CambridgeUS}
\makeatletter
\makeatother
\usetikzlibrary{shapes,backgrounds}
\tikzstyle{cblue}=[circle, draw, thin,fill=cyan!20, scale=0.8]
\tikzstyle{qgre}=[rectangle, draw, thin,fill=green!20, scale=0.8]
\tikzstyle{rpath}=[ultra thick, red, opacity=0.4]
\tikzstyle{legend_isps}=[rectangle, rounded corners, thin,
                       fill=gray!20, text=blue, draw]
\usetikzlibrary{decorations.pathreplacing}
\tikzset{text/.default=}
%\tikzset{text/.align=0}
\tikzstyle{every picture}+=[remember picture]
\tikzstyle{na} = [baseline=-.5ex]
\setbeamertemplate{itemize item}{\color{black}$\bullet$}
\setbeamertemplate{itemize subitem}{\color{black}$\bullet$}

\usetikzlibrary{shapes}
\usetikzlibrary{positioning}
\usetikzlibrary{automata}
\usepackage{amsmath,amssymb,amsfonts,amsthm}
\setbeamercovered{invisible} %% <--- I ADDED THIS
\newcommand{\red}{\textcolor{red}}
\newcommand{\blue}{\textcolor{blue}}
\newcommand{\purple}{\textcolor{purple}}
\newcommand{\brown}{\textcolor{brown}}
\newcommand{\cyan}{\textcolor{cyan}}
\newcommand{\real}{\ensuremath{\mathbb{R}}}
\newcommand{\y}{\ensuremath{\mathbf{y}}}
\newcommand{\black}{\color{black}}
\newcommand{\btheta}{\boldsymbol{\theta}}
\newcommand{\green}{\color{green}}
\newcommand{\word}[1]{\green{\textit{#1}\ }\black}
\newcommand{\lb}{\linebreak}
\newcommand{\vitem}{\item}
\newtheorem{com}{Comment}
\newtheorem{lem} {Lemma}
\newtheorem{prop}{Proposition}
\newtheorem{thm}{Theorem}
\newtheorem{defn}{Definition}
\newtheorem{cor}{Corollary}
\newtheorem{obs}{Observation}
\setcounter{tocdepth}{1}

\definecolor{UBCblue}{rgb}{0.04706, 0.13725, 0.26667}
\definecolor{UBCgray}{rgb}{0.3686, 0.5255, 0.6235}
\colorlet{verylightgray}{gray!10}
\setbeamercolor{palette primary}{bg=UBCblue,fg=white}
\setbeamercolor{palette secondary}{bg=darkgray,fg=white}
\setbeamercolor{palette tertiary}{bg=UBCblue,fg=white}
\setbeamercolor{palette quaternary}{bg=UBCblue,fg=white}
\setbeamercolor{structure}{fg=UBCblue} % itemize, enumerate, etc
\setbeamercolor{section in toc}{fg=UBCblue} % TOC sections
\setbeamercolor{subsection in head/foot}{bg=darkgray,fg=white}
\setbeamercolor{frametitle}{fg=UBCblue}
\setbeamercolor{title}{fg=UBCblue, bg=verylightgray}

\title[Class 23]{Introduction to Social Data Analytics: \\
Class 23}
\author[]{}
\institute[UCSD]{}
\date[]{}

\begin{document}

\frame{\titlepage}

\begin{frame}
\frametitle{Today: Regression in R}
By the end of today's lecture, you should be able to:
\begin{itemize}
	\item Perform regression analysis to determine linear relationships between variables
	\item Interpret coefficient estimates and add best fit lines to scatter plots of the data
\end{itemize} \bigskip
We'll work with three datasets: nba.csv, florida.csv, and face.csv.
\end{frame}

\begin{frame}[fragile]
 \frametitle{Review of regression basics}
 \pause
\begin{itemize} \itemsep1em
\item Regression: finds the \alert{best fit} line between $k$ independent and one dependent variables.
\item One independent variable case: $y_i = \beta_0 + \beta_1x_i + \epsilon_i$ \pause
\item Produces \alert{estimates}: $\hat{\beta_0}$ and $\hat{\beta_1}$ 
\item \alert{Interpretation:} \pause
	\begin{itemize}
		\item $\hat{\beta_1}$: a one-unit increase in $x$-units is associated with a $\hat{\beta_1}$ average increase in $y$-units. \pause
		\item $\hat{\beta_0}$: \alert{expected} value of $y$ when $x=0$. 
	\end{itemize}
\end{itemize}
\end{frame}

\begin{frame}[fragile]
 \frametitle{Some useful terms}
\begin{itemize} \itemsep1em
\item \alert{Fitted values}, \alert{predicted values}, \alert{$\hat{y}$}:
\begin{itemize}
\item predicted $y$ when $x$ is set to a certain value.
\item predicted $y$ for $x$'s observed in the dataset
\end{itemize}
\item \alert{Residuals}
\begin{itemize}
\item Difference between \alert{predicted} and \alert{actual} value of $y$.
\end{itemize}
\item Let's return to the NBA dataset.
\end{itemize}
\end{frame}

\begin{frame}[fragile]
 \frametitle{Height and Blocks}
 \begin{shaded}
\begin{verbatim}
# Make scatter plot
plot(nba$height, nba$blocks, 
	  xlab = "Height", 
	  ylab = "Blocks",
	  pch = 16, 
	  main = "Height of NBA Players and Blocks")
\end{verbatim}
\end{shaded}
\end{frame}

\begin{frame}[fragile]
 \frametitle{Height and Blocks}
\begin{center}
\includegraphics[scale=.6]{NBAscatter.png}
\end{center}
\end{frame}

\begin{frame}[fragile]
 \frametitle{Estimating Height and Blocks Relationship}
 \begin{shaded}
 \begin{small}
\begin{verbatim}
# What is the correlation between height and blocks?
cor(nba$height, nba$blocks)
[1] 0.8797669

# Fit a regression line of the form:
#   blocks = b0 + b1 * height
fit <- lm(blocks ~ height, data=nba)

# Check the results
summary(fit)

\end{verbatim}
\end{small}
\end{shaded}
\end{frame}

\begin{frame}[fragile]
 \frametitle{Estimating Height and Blocks Relationship}
  \begin{shaded}
 \begin{small}
\begin{verbatim}
summary(fit)

Call:
lm(formula = blocks ~ height, data = nba)

Residuals:
     Min       1Q   Median       3Q      Max 
-0.39475 -0.24761 -0.04251  0.09506  0.97341 

Coefficients:
            Estimate Std. Error t value Pr(>|t|)    
(Intercept) -7.73418    1.23275  -6.274 2.04e-05 ***
height       0.10796    0.01559   6.924 7.05e-06 ***
---
Signif. codes:  0 *** 0.001 ** 0.01 * 0.05 . 0.1  1

Residual standard error: 0.3401 on 14 degrees of freedom
Multiple R-squared:  0.774,	Adjusted R-squared:  0.7578 
F-statistic: 47.94 on 1 and 14 DF,  p-value: 7.047e-06

\end{verbatim}
\end{small}
\end{shaded}
\end{frame}

\begin{frame}[fragile]
 \frametitle{Add best fit line and players' names to plot}
  \begin{shaded}
 \begin{small}
\begin{verbatim}
# Now add the best fit line to the plot
abline(fit)

# Now add the players' names
text(nba$height, nba$blocks+.1, nba$name, cex=0.6)

\end{verbatim}
\end{small}
\end{shaded}
\end{frame}

\begin{frame}[fragile]
 \frametitle{Height and Blocks}
\begin{center}
\includegraphics[scale=.6]{NBAScatter2.png}
\end{center}
\end{frame}

\begin{frame}[fragile]
 \frametitle{Election 2000, Butterfly Ballot}
\begin{itemize}
\item Suppose we wanted to predict third-party votes in Florida 
\item Use 1996 election (Perot) to predict 2000 election (Buchanan)
\end{itemize}
\begin{shaded}
\begin{small}
\begin{verbatim}
head(florida)
    county Clinton96 Dole96 Perot96 Bush00 Gore00 Buchanan00
1  Alachua     40144  25303    8072  34124  47365        263
2    Baker      2273   3684     667   5610   2392         73
3      Bay     17020  28290    5922  38637  18850        248
4 Bradford      3356   4038     819   5414   3075         65
5  Brevard     80416  87980   25249 115185  97318        570
6  Broward    320736 142834   38964 177323 386561        788
\end{verbatim}
\end{small}
\end{shaded}
\end{frame}

\begin{frame}[fragile]
 \frametitle{Do Votes for Perot Predict Votes for Buchanan?}
 Your task is to do the following:
 \begin{itemize} \itemsep1em
 	\item Plot votes for Perot96 vs votes for Buchanan00 (both third party presidential candidates)
 	\item Fit a best fit line through your scatterplot
 	\item What seems off?
\end{itemize}
\end{frame}

\begin{frame}[fragile]
 \frametitle{Do Votes for Perot Predict Votes for Buchanan?}
 \begin{shaded}
\begin{small}
\begin{verbatim}
fit2 <- lm(Buchanan00 ~ Perot96, data = florida)
summary(fit2)

Call:
lm(formula = Buchanan00 ~ Perot96, data = florida)

Residuals:
    Min      1Q  Median      3Q     Max 
-612.74  -65.96    1.94   32.88 2301.66 

Coefficients:
            Estimate Std. Error t value Pr(>|t|)    
(Intercept)  1.34575   49.75931   0.027    0.979    
Perot96      0.03592    0.00434   8.275 9.47e-12 ***

\end{verbatim}
\end{small}
\end{shaded}
\end{frame}

\begin{frame}[fragile]
 \frametitle{What seems off?}
\begin{center}
\includegraphics[scale=.4]{Florida9600.png}
\end{center}
\end{frame}

\begin{frame}[fragile]
 \frametitle{Which county in Florida is the outlier?}
 \begin{shaded}
\begin{verbatim}
florida$county[resid(fit2) == max(resid(fit2))]
[1] PalmBeach
\end{verbatim}
\end{shaded}
\begin{center}
\includegraphics[scale=.4]{palmballot}
\end{center}
\end{frame}


\begin{frame}[fragile]
 \frametitle{Regression with experiments}
\begin{itemize}
\item Experiment: Does \alert{facial appearance} predict \alert{vote outcomes}?
\item Todorov et al (2005): 
\begin{itemize}
\item Quickly show picture of candidates to subjects
\item Ask them to rate on ``competence''
\end{itemize}
\end{itemize}
\begin{center}
\includegraphics[scale=0.75]{faces.png}
\end{center}
\end{frame}

\begin{frame}[fragile]
 \frametitle{The data}
\begin{shaded}
\begin{verbatim}
head(face)
\end{verbatim}
\end{shaded}
\begin{scriptsize}
\begin{verbatim}
  year state    winner     loser w.party l.party    d.comp    r.comp d.votes r.votes
1 2000    CA Feinstein  Campbell       D       R 0.5645676 0.4354324 5790154 3779325
2 2000    DE    Carper      Roth       D       R 0.3419122 0.6580878  181387  142683
3 2000    FL    Nelson  McCollum       D       R 0.6123680 0.3876320 2987644 2703608
4 2000    GA    Miller Mattingly       D       R 0.5415328 0.4584672 1390428  933698
5 2000    HI     Akaka   Carroll       D       R 0.6802323 0.3197677  251130   84657
6 2000    IN     Lugar   Johnson       R       D 0.3205024 0.6794976  684242 1419629
\end{verbatim}
\end{scriptsize} \bigskip
\texttt{d.comp} is the fraction of subjects who think the democratic candidate appears more competent. 
\end{frame}

\begin{frame}[fragile]
 \frametitle{Does Facial Appearance Predict Vote Share?}
 \begin{shaded}
 \begin{small}
\begin{verbatim}
#define vote share for Dems and Reps
face$d.share <- face$d.votes/(face$d.votes + face$r.votes)
face$r.share <- face$r.votes/(face$d.votes + face$r.votes)
face$diff.share <- face$d.share - face$r.share
\end{verbatim}
\end{small}
\end{shaded}
\begin{itemize}
	\item Add a plot of Democratic candidate's competence score on the x-axis and Democratic margin in vote share on the y-axis, and label as appropriate.
	\item Regress the model diff.share = $\beta_0 + \beta_1 *$ d.comp
	\item Add a trendline to your plot
	\item Is there a causal effect of appearance on voting outcome?
\end{itemize}
\end{frame}

\begin{frame}[fragile]
 \frametitle{Does Facial Appearance Predict Vote Share?}
\begin{center}
\includegraphics[scale=.4]{CompetenceandVoteShare.pdf}
\end{center}
\end{frame}

\begin{frame}[fragile]
 \frametitle{How do we interpret these coefficients?}
\begin{small}
\begin{verbatim}
fit3 <- lm(diff.share ~ d.comp, data = face)
summary(fit3)

Call:
fit3 <- lm(formula = diff.share ~ d.comp, data = face)

Residuals:
     Min       1Q   Median       3Q      Max 
-0.67487 -0.16600  0.01399  0.17741  0.74297 

Coefficients:
            Estimate Std. Error t value Pr(>|t|)    
(Intercept) -0.31223    0.06596  -4.733 6.24e-06 ***
d.comp       0.66038    0.12718   5.193 8.85e-07 ***
---
\end{verbatim}
\end{small}
\end{frame}

\begin{frame}[fragile]
 \frametitle{Does Facial Appearance Predict Vote Share?}
\begin{center}
\includegraphics[scale=.4]{faces2.png}
\end{center}
\end{frame}

\end{document}











