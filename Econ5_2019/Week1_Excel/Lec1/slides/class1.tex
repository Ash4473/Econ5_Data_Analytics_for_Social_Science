\documentclass[11pt]{beamer}
\mode<presentation>
%\documentclass[handout,compress]{beamer}
\usepackage{beamerthemedefault}
\usepackage{graphicx}
\usepackage{hyperref}
\usepackage{subfigure}
\usepackage{color}
\usepackage{multicol}
\usepackage{bm} 
\usepackage{tikz}
\usepackage{listliketab}

\usetheme{CambridgeUS}
\makeatletter
\makeatother
\usetikzlibrary{shapes,backgrounds}
\tikzstyle{cblue}=[circle, draw, thin,fill=cyan!20, scale=0.8]
\tikzstyle{qgre}=[rectangle, draw, thin,fill=green!20, scale=0.8]
\tikzstyle{rpath}=[ultra thick, red, opacity=0.4]
\tikzstyle{legend_isps}=[rectangle, rounded corners, thin,
                       fill=gray!20, text=blue, draw]
\usetikzlibrary{decorations.pathreplacing}
\tikzset{text/.default=}
%\tikzset{text/.align=0}
\tikzstyle{every picture}+=[remember picture]
\tikzstyle{na} = [baseline=-.5ex]
\setbeamertemplate{itemize item}{\color{black}$\bullet$}
\setbeamertemplate{itemize subitem}{\color{black}$\bullet$}

\usetikzlibrary{shapes}
\usetikzlibrary{positioning}
\usetikzlibrary{automata}
\usepackage{amsmath,amssymb,amsfonts,amsthm}
\setbeamercovered{invisible}
\newcommand{\red}{\textcolor{red}}
\newcommand{\blue}{\textcolor{blue}}
\newcommand{\purple}{\textcolor{purple}}
\newcommand{\brown}{\textcolor{brown}}
\newcommand{\cyan}{\textcolor{cyan}}
\newcommand{\real}{\ensuremath{\mathbb{R}}}
\newcommand{\y}{\ensuremath{\mathbf{y}}}
\newcommand{\black}{\color{black}}
\newcommand{\btheta}{\boldsymbol{\theta}}
\newcommand{\green}{\color{green}}
\newcommand{\word}[1]{\green{\textit{#1}\ }\black}
\newcommand{\lb}{\linebreak}
\newtheorem{com}{Comment}
\newtheorem{lem} {Lemma}
\newtheorem{prop}{Proposition}
\newtheorem{thm}{Theorem}
\newtheorem{defn}{Definition}
\newtheorem{cor}{Corollary}
\newtheorem{obs}{Observation}
\setcounter{tocdepth}{1}

\definecolor{UBCblue}{rgb}{0.04706, 0.13725, 0.26667}
\definecolor{UBCgray}{rgb}{0.3686, 0.5255, 0.6235}
\colorlet{verylightgray}{gray!10}
\setbeamercolor{palette primary}{bg=UBCblue,fg=white}
\setbeamercolor{palette secondary}{bg=darkgray,fg=white}
\setbeamercolor{palette tertiary}{bg=UBCblue,fg=white}
\setbeamercolor{palette quaternary}{bg=UBCblue,fg=white}
\setbeamercolor{structure}{fg=UBCblue} % itemize, enumerate, etc
\setbeamercolor{section in toc}{fg=UBCblue} % TOC sections
\setbeamercolor{subsection in head/foot}{bg=darkgray,fg=white}
\setbeamercolor{frametitle}{fg=UBCblue}
\setbeamercolor{title}{fg=UBCblue, bg=verylightgray}

\title[Class 1]{Introduction to Social Data Analytics \\
\bigskip Class 1}
\author[Kaushik]{Arushi Kaushik}
\institute[UCSD]{Department of Economics \\ UCSD}
\date{}


% Edit the above section as well as entering a few pieces of information below:
% Office hours
% A slide about yourself
% A slide about how you use data analysis in your research
% Update survey link if necessary


\begin{document}

\frame{\titlepage}

\begin{frame}
 \frametitle{Why take this class?}
  \begin{itemize}
\item Social science is experiencing a \alert{revolution} \pause
\item Social data \alert{EXPLOSION} available from:
\begin{itemize}
\item governments
\item non-profits
\item businesses
\item citizens 
\end{itemize}\pause
\item Better \alert{computing power} and \alert{statistical} techniques to analyze it
\item Requires knowledge of \alert{social science}, \alert{computer programming} and \alert{statistics} to be useful! 
\pause
\item The short answer: \textbf{better career prospects}
\end{itemize}
\end{frame}

\begin{frame}
\frametitle{Learning objectives and class goals}
By the end of this course, you should be able to:
\begin{itemize}
\item Analyze data to solve \alert{real world problems}
\item Conduct basic operations in \alert{Excel, Stata,} and \alert{R} 
\item Identify resources for \alert{further learning}
\item Feel \alert{inspired} and \alert{empowered} to work with data
\end{itemize}
\end{frame}

\begin{frame}
\frametitle{What you can expect of me}
\begin{itemize}
\item Enthusiasm: I am \alert{passionate} about data analytics \pause
\item Timeliness: class will start \alert{on time}, questions answered in 24 hours \pause
\item Helpfulness: \alert{your success} is my goal and my priority
\begin{itemize}
	\item[-] I will happily answer any and all questions
	\item[-] My office hour is Wednesdays 2-3 pm at Room No. 124, Economics Building
	\item[-] My email address: arkaushi@ucsd.edu
\end{itemize}
\end{itemize}
\end{frame}

\begin{frame}
\frametitle{My expectations of you}
\begin{itemize}
\item Adopt a \alert{growth mindset:} \uncover
{work hard, seek out resources, believe in yourself} \pause
\item \alert{Interrupt} me and ask questions! Especially \textit{when} I make mistakes \pause
\item Use \alert{Piazza} for all course content-related questions \pause
\item \alert{Actively participate}:
	\begin{itemize}
	\item[-] in class: lecture will be interactive
	\item[-] on Piazza: help your classmates
	\item[-] on assignments: working in groups okay (and encouraged) but copying not
	\end{itemize}
\end{itemize}
\bigskip
\pause
Let's start with introductions.
\end{frame}

\begin{frame}
\frametitle{Who am I?}
\begin{center}
 \includegraphics[scale=.30]{images/arushi2.png} 
\end{center}
\end{frame}

\begin{frame}
\begin{center}
\frametitle{Data analytics in my research}
 \includegraphics[scale=.35]{images/BhopalGasDisaster.png} 
\end{center}
\end{frame}


\begin{frame}
\begin{center}
\frametitle{Data analytics in my research}
 \includegraphics[scale=.15]{images/Cancer_Cohort.jpg} 
\end{center}
\end{frame}


\begin{frame}
\frametitle{Introduce yourself to your neighbor}
\begin{itemize}
\item Good places to start:
	\begin{itemize}
	\item[-] Name
	\item[-] Major
	\item[-] Hometown
	\end{itemize}
\item I like to share something \alert{memorable}
\item Exchange phone number and/or email 
\pause
\bigskip
\item I'd like to meet all of you, too!
\end{itemize}
\end{frame}

\begin{frame}
\frametitle{How do we get started?}
\uncover
{\begin{center}
Hilary Mason (Chief Data Scientist at Bitly) \\
\includegraphics[scale=.15]{images/HilaryMason.png}
\end{center}
}
\bigskip
\uncover<+->
{What you need to do data science:}
\begin{enumerate}
\item A question
\item A dataset
\item Something to analyze it with
\end{enumerate}

\end{frame}


\begin{frame}
\frametitle{1. A question}
Good questions could come from:
\begin{itemize}
\item Tech Firms/Startups
\item Social media
\item Government policy 
\item Academic research
\item Political campaigns $\hdots$ and much more!
\end{itemize}
\end{frame}

\begin{frame}
\frametitle{2. A dataset}
It's easy to find datasets online:
\begin{itemize}
\item data.gov
\item census.gov/data
\item ipums.org
\item data.worldbank.org
\item registry.opendata.aws
\item Our TritonEd page for data used in this course \pause
\end{itemize}
We will begin exploring data next class.
\end{frame}

\begin{frame}
\frametitle{3. Something to analyze data with}
How will we work with the data? So many options....\\
\bigskip
\uncover
{Excel, SAS, SPSS, Stata, R, Python, Matlab, Gauss, C, Java, SciLab...}\\
\bigskip
Our focus: \alert{Excel}, \alert{Stata}, \alert{R} \pause
\begin{itemize}
\item Why only these three?
\begin{itemize}
\item \alert{Important} for the social sciences at UCSD
\item Highly requested by \alert{employers}
\end{itemize} \pause
\item Why not just focus on one?
\begin{itemize}
\item You will \alert{shift} between softwares in your career
\item We'll focus on the \alert{main takeaways} and demonstrate how they apply to each
\item You will learn how to \alert{further} your knowledge independently
\end{itemize}
\end{itemize}
\end{frame}

\begin{frame}
\frametitle{3.  Something to analyze data with}
Main takeaways:
\begin{itemize}
\item How to \alert{import} data
\item How to \alert{automate} routines and \alert{reproduce} them
\item How to \alert{visualize} data 
\item How to conduct \alert{regressions}
\item How to use \alert{logic} in \alert{if statements}
\item How to use \alert{for loops}
\item How to write \alert{functions}
\end{itemize}
\uncover
{All the while better understanding \alert{social science questions}!}
\end{frame}


\begin{frame}\frametitle{The Organizing Framework}
\begin{enumerate}
\item<1-> Unit 1: \alert{Excel}
\uncover<1->{\begin{itemize}
\item Introduction to data structure
\item Functions and plotting in Excel
\end{itemize}}
\pause
\item<1-> Unit 2: \alert{Stata}
\begin{itemize}
\item Introduction to scripting and reproducibility
\item Descriptive statistics
\item If statements 
\item Graphics and regression 
\item Data wrangling
\end{itemize}
\pause
\item<1-> Unit 3: \alert{R}
\begin{itemize}
\item Introduction to scripting in R
\item Visualizing data 
\item If statements and for loops 
\item Regression and functions
\item Data wrangling
\end{itemize}
\end{enumerate}
\end{frame}

\begin{frame}
\frametitle{Assignments}

\begin{itemize}
\item[-] \alert{Four Problem Sets}: (40\% of grade)
\begin{itemize}
\item[-] One Excel, one Stata, and two R
\item[-] Turn in commented code that can run 
\item[-] List names of people you worked with on first sheet of solutions \pause
\end{itemize}
\item[-] \alert{Midterm}: (20\%) 
\begin{itemize}
	\item[-] computer-based assessment of data analysis in Excel and Stata \pause
\end{itemize}
\item[-] \alert{Final Project}: (30\%) 
\begin{itemize}
\item[-] Updates due throughout quarter with problem sets \pause
\end{itemize}
\item[-] \alert{Pre-class exercises}: Due midnight before each class (5\%) 
\begin{itemize}
\item[-] Will help prepare you for in class activities \pause
\end{itemize}
\item[-] \alert{Participation}: (5\%)
\begin{itemize}
\item[-] Lecture and section \alert{attendance required}
\end{itemize}
\end{itemize}
\end{frame}

\begin{frame}
\frametitle{Course Materials}
\begin{itemize}
\item Before each class, you will be expected to submit brief exercises
\begin{itemize}
	\item The best way to improve your analytics skills is to \alert{practice}!
	\item These exercises will prep you for in-class activities \pause
\end{itemize}
\item Course materials listed in the syllabus will help with the exercises
\begin{itemize}
\item Readings
\item Youtube videos
\item When we get to R, \textit{A First Course in Quantitative Social Science} by Kosuke Imai \pause
\end{itemize}
\item Datasets, problem sets, and more will be posted on \alert{TritonEd} \pause
\item Questions on course content should be asked on \alert{Piazza}
\end{itemize}
\end{frame}

\begin{frame}
\frametitle{Software you will need}
\begin{itemize}
\item[-] \alert{Excel}
\begin{itemize}
\item[-] Install before next class (free campus license)
\item[-] Available on most campus computers
\end{itemize}
\item[-] \alert{Stata}
\begin{itemize}
\item[-] Normally expensive
\item[-] We will provide a campus license
\end{itemize}
\item[-] \alert{R}
\begin{itemize}
\item[-] Free!  
\item[-] I will show you how to download in the second half of the course
\end{itemize}
\end{itemize}
\end{frame}

\begin{frame}
\frametitle{Some housekeeping}
What does \alert{academic integrity} mean to you?\\
\pause
\bigskip
From the Academic Integrity Office: ``Cheating occurs when a student attempts to get academic credit in a way that is dishonest, disrespectful, irresponsible, untrustworthy, or unfair.''\\
\pause
\bigskip
We will take academic integrity seriously. 
\end{frame}

\begin{frame}
\frametitle{I want to know about you.}

In the spirit of data science....
\bigskip
\begin{center}
https://bit.ly/2q1mUdg % Update this link if needed
\end{center}
\bigskip
\bigskip
Remember to submit your pre class exercise (via TritonEd) before next class!

\end{frame}

\end{document}









