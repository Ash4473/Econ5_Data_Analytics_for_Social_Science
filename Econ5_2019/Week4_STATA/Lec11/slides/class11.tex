\documentclass[11pt]{beamer}
\mode<presentation>
%\documentclass[handout,compress]{beamer}
\usepackage{beamerthemedefault}
\usepackage{graphicx}
\usepackage{tcolorbox}
\usepackage{hyperref}
\usepackage{subfigure}
\usepackage{color}
\usepackage{multicol}
\usepackage{bm} 
\usepackage{tikz}
\usepackage{listliketab}
\usepackage{tabto}

\usetheme{CambridgeUS}
\makeatletter
\makeatother
\usetikzlibrary{shapes,backgrounds}
\tikzstyle{cblue}=[circle, draw, thin,fill=cyan!20, scale=0.8]
\tikzstyle{qgre}=[rectangle, draw, thin,fill=green!20, scale=0.8]
\tikzstyle{rpath}=[ultra thick, red, opacity=0.4]
\tikzstyle{legend_isps}=[rectangle, rounded corners, thin,
fill=gray!20, text=blue, draw]
\usetikzlibrary{decorations.pathreplacing}
\tikzset{text/.default=}
%\tikzset{text/.align=0}
\tikzstyle{every picture}+=[remember picture]
\tikzstyle{na} = [baseline=-.5ex]
\setbeamertemplate{itemize item}{\color{black}$\bullet$}
\setbeamertemplate{itemize subitem}{\color{black}$\bullet$}

\usetikzlibrary{shapes}
\usetikzlibrary{positioning}
\usetikzlibrary{automata}
\usepackage{amsmath,amssymb,amsfonts,amsthm}
\setbeamercovered{invisible} %% <--- I ADDED THIS
\newcommand{\red}{\textcolor{red}}
\newcommand{\blue}{\textcolor{blue}}
\newcommand{\orange}{\textcolor{orange}}
\newcommand{\purple}{\textcolor{violet}}
\newcommand{\brown}{\textcolor{brown}}
\newcommand{\cyan}{\textcolor{cyan}}
\newcommand{\real}{\ensuremath{\mathbb{R}}}
\newcommand{\y}{\ensuremath{\mathbf{y}}}
\newcommand{\black}{\color{black}}
\newcommand{\btheta}{\boldsymbol{\theta}}
\definecolor{ao}{rgb}{.635, .766, .365}
\newcommand{\greenn}{\textcolor{ao}}
\newcommand{\word}[1]{\green{\textit{#1}\ }\black}
\newcommand{\lb}{\linebreak}
\newtheorem{com}{Comment}
\newtheorem{lem} {Lemma}
\newtheorem{prop}{Proposition}
\newtheorem{thm}{Theorem}
\newtheorem{defn}{Definition}
\newtheorem{cor}{Corollary}
\newtheorem{obs}{Observation}
\setcounter{tocdepth}{1}

\definecolor{UBCblue}{rgb}{0.04706, 0.13725, 0.26667}
\definecolor{UBCgray}{rgb}{0.3686, 0.5255, 0.6235}
\colorlet{verylightgray}{gray!10}
\setbeamercolor{palette primary}{bg=UBCblue,fg=white}
\setbeamercolor{palette secondary}{bg=darkgray,fg=white}
\setbeamercolor{palette tertiary}{bg=UBCblue,fg=white}
\setbeamercolor{palette quaternary}{bg=UBCblue,fg=white}
\setbeamercolor{structure}{fg=UBCblue} % itemize, enumerate, etc
\setbeamercolor{section in toc}{fg=UBCblue} % TOC sections
\setbeamercolor{subsection in head/foot}{bg=darkgray,fg=white}
\setbeamercolor{frametitle}{fg=UBCblue}
\setbeamercolor{title}{fg=UBCblue, bg=verylightgray}

\title[Class 11]{Introduction to Social Data Analytics \\
	\bigskip Class 11}
\author[]{}
\institute[UCSD]{}
\date{}

% update title block above

\begin{document}

\frame{\titlepage}

\begin{frame}
\frametitle{Today: Regression in Stata}
By the end of today's lecture, you should be able to:
\begin{itemize} \itemsep1em
	\item Conduct basic regression analysis in Stata using \texttt{reg}
	\item Explain why one must be careful with linear form assumptions and out of sample extrapolation
	\item Distinguish causal effects from correlations between variables, and describe how naive regression is useful 
	\item Analyze regression results and interpret key elements such as coefficient estimates and variance
	\item Construct a best fit line in a scatterplot and identify the slope, intercept, and residuals
\end{itemize}
\end{frame}

\begin{frame}
\frametitle{Suppose we want to know how Y varies with X in a population.}
We might \alert{model} the outcome, Y, as a function of the predictor, X: \\
$$Y_i = f(X_i)$$ \\ \bigskip
If we assume the relationship is \alert{linear}, we can write our model as: \\
$$Y_i = \beta_0 + \beta_1X_i + \varepsilon_i $$ \\ \bigskip
Our job is to \alert{estimate} $\beta_0$ and $\beta_1$ using data. 
\end{frame}

\begin{frame}
\frametitle{Let's break down the linear model.}
$$Y_i = \greenn{\beta_0} + \red{\beta_1}X_i + \orange{\varepsilon_i} $$ 
\begin{center}
	Y-value = \greenn{Intercept} + \red{Slope} * X-value + \orange{error} \\ \bigskip
	\includegraphics[height = 0.5\textheight]{images/largescatter.png}
\end{center}
\end{frame}

\begin{frame}
\frametitle{We use `hats' to denote coefficient estimates.}
$$Y_i = \beta_0 + \beta_1X_i + \varepsilon_i $$ \\ \bigskip
Often we don't have data for the entire population, so we cannot calculate the exact population parameters $\beta_0$ and $\beta_1$. \\ \bigskip \pause
Instead, we use a representative sample to \alert{estimate} them: \\ \bigskip
\begin{center}
	$\hat{\beta}_0$: estimate of the intercept, $\beta_0$  \\
	$\hat{\beta}_1$: estimate of the coefficient, $\beta_1$  
\end{center} \pause \bigskip
$\hat{\beta}_0$ and $\hat{\beta}_1$ are the parameter estimates that \alert{best fit} the sample data. 
\end{frame}

\begin{frame}
\frametitle{How do we estimate $\beta_0$ and $\beta_1$?}
Primary tool: \alert{Ordinary Least Squares (OLS)} \\ \bigskip
Choose $\hat{\beta}_0$ and $\hat{\beta}_1$ that minimize the sum of $\varepsilon_i^2$. 
\begin{center}
	\includegraphics[height = 0.5\textheight]{images/bestfit.png}
\end{center} \pause
Computing $\hat{\beta}_0$ and $\hat{\beta}_1$ manually can take a very long time...but regression in Stata takes only a few seconds!
\end{frame}

\begin{frame}
\frametitle{We might ask...}
\begin{center}
	\includegraphics[height = 0.8\textheight]{images/nbadata2.png}
\end{center}
\end{frame}

\begin{frame}
\frametitle{Scatter plot of the data}
\begin{center}
	\includegraphics[height = 0.8\textheight]{images/nbascatter.png}
\end{center}
\end{frame}

\begin{frame}
\frametitle{Regression \textit{can} tell us...}
\begin{center}
	\includegraphics[height = 0.8\textheight]{images/regression1.png}
\end{center}
\end{frame}

\begin{frame}
\frametitle{Linear regression allows us to construct a line of best fit}
\begin{center}
	\includegraphics[height = 0.8\textheight]{images/regression2.png}
\end{center}
\end{frame}

\begin{frame}
\frametitle{How do we interpret the \alert{slope} estimate, $\hat{\beta}_1$?}
\begin{tcolorbox}[]
 	[\purple{Y}] changes by [\red{$\hat{\beta}_1$}] [\purple{Y} units] for every one [\cyan{X} unit] increase in [\cyan{X}]... 
 	\begin{flushright}
 		...\textit{on average, all else equal.}
 	\end{flushright}
\end{tcolorbox}
In our NBA example:
\begin{itemize}
	\item \red{$\hat{\beta}_1$} = 0.108
	\item \purple{Y} = ``number of blocks per game"
	\item \cyan{X} = ``height in inches"
\end{itemize}  \pause \bigskip
``\purple{An NBA player's blocks per game} increases by \red{0.108} \purple{blocks} for every one \cyan{inch} increase in \cyan{height} on average, all else equal."
\end{frame}

\begin{frame}
\frametitle{How do we interpret the \alert{intercept} estimate, $\hat{\beta}_0$?}
\begin{tcolorbox}[]
	When [\cyan{X}] is zero [\cyan{X} units],  [\purple{Y}] is [\greenn{$\hat{\beta}_0$}] [\purple{Y} units]... 
	\begin{flushright}
		...\textit{on average, all else equal.}
	\end{flushright}
\end{tcolorbox}
In our NBA example:
\begin{itemize}
	\item \greenn{$\hat{\beta}_0$} = -7.734
	\item \purple{Y} = ``number of blocks per game"
	\item \cyan{X} = ``height in inches"
\end{itemize}  \pause \bigskip
``When \cyan{an NBA player's height} is zero \cyan{inches}, his \purple{blocks per game} is \greenn{-7.734} \purple{blocks} on average, all else equal."
\end{frame}

\begin{frame}
\frametitle{If Tony Parker is 74" tall, what is his predicted blocks/game?}
\begin{columns}
	\begin{column}{0.5\textwidth}
		\includegraphics[width = \textwidth]{images/tonyparker.png}
	\end{column}
	\begin{column}{0.5\textwidth}
		$\hat{Y_i} = \greenn{-7.734} + \red{0.108}X_i $ \pause \\
		$\hat{Y_i} = \greenn{-7.734} + \red{0.108}(74) = 0.258 $ \\ \pause \bigskip
		If his actual blocks/game average was 0.1, what's the model error (residual)? \\ \pause \bigskip
		$\orange{Error_i} = Actual_i - Predicted_i$ \\ 
		$\orange{\varepsilon_i} = Y_i - \hat{Y_i}$ \\ \pause
		$\orange{\varepsilon_i} = 0.1 - 0.258 = -0.158$
	\end{column}
\end{columns}
\end{frame}

\begin{frame}
\frametitle{A cautionary tale: out-of-sample extrapolation}
\begin{center}
	\includegraphics[height = 0.8\textheight]{images/extrapolate.png}
\end{center}
\end{frame}

\begin{frame}
\frametitle{Can we (accurately) do the following with our model:}
\begin{itemize}
	\item Predict blocks for an NBA player 68 inches in height or shorter? \pause
	\item[] \tabto{1cm} No, 68 inches is outside the domain of our data. \pause
	\item[]
	\item Predict blocks for a \textit{college} basketball player of 75 inches? \pause
	\item[] \tabto{1cm} No, our results are valid only for NBA players. \pause
	\item[]
	\item Predict how many blocks an NBA player would get if he wore shoes that raised his height by 5 inches? \pause
	\item[] \tabto{1cm} No, our model estimates apply only to \textit{natural} height.
\end{itemize} \bigskip \pause
\begin{center}
	Be careful with \alert{extrapolation} and \alert{external validity}!
\end{center}
\end{frame}

\begin{frame}
\frametitle{Open class11.do in Stata}
\begin{center}
	\includegraphics[width = 0.85\textwidth]{images/stata1.png}
\end{center}
\begin{itemize} \pause
	\item What does the Coef. tell us?
	\item What does the Std. Err. tell us?
	\item Are the coefficients statistically significant? Are they economically significant?
\end{itemize}
\end{frame}

\begin{frame}
\frametitle{Generating predicted values}
\texttt{predict yhat \\ scatter yhat height}
\begin{center}
	\includegraphics[width = 0.6\textwidth]{images/stata2.png}
\end{center}
\end{frame}

\begin{frame}
\frametitle{Generating residuals}
\texttt{predict resid, residuals \\ histogram resid, width(0.1) frequency normal}
\begin{center}
	\includegraphics[width = 0.6\textwidth]{images/stata3.png}
\end{center}
\end{frame}

\begin{frame}
\frametitle{A bonus cautionary tale: Anscombe's Quartet}
\begin{center}
	\includegraphics[height = 0.85\textheight]{images/anscombe1.png}
\end{center}
\end{frame}

\begin{frame}
\frametitle{A bonus cautionary tale: Anscombe's Quartet}
\begin{center}
	\includegraphics[width = \textwidth]{images/anscombe2.png}
	Lesson learned: \textbf{Regression output does not tell the full story!} \\ \medskip
\end{center}
\end{frame}

\end{document}