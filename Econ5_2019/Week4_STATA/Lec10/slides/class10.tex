\documentclass[11pt]{beamer}
\mode<presentation>
%\documentclass[handout,compress]{beamer}
\usepackage{beamerthemedefault}
\usepackage{graphicx}
\usepackage{hyperref}
\usepackage{subfigure}
\usepackage{color}
\usepackage{multicol}
\usepackage{bm} 
\usepackage{tikz}
\usepackage{listliketab}

\usetheme{CambridgeUS}
\makeatletter
\makeatother
\usetikzlibrary{shapes,backgrounds}
\tikzstyle{cblue}=[circle, draw, thin,fill=cyan!20, scale=0.8]
\tikzstyle{qgre}=[rectangle, draw, thin,fill=green!20, scale=0.8]
\tikzstyle{rpath}=[ultra thick, red, opacity=0.4]
\tikzstyle{legend_isps}=[rectangle, rounded corners, thin,
fill=gray!20, text=blue, draw]
\usetikzlibrary{decorations.pathreplacing}
\tikzset{text/.default=}
%\tikzset{text/.align=0}
\tikzstyle{every picture}+=[remember picture]
\tikzstyle{na} = [baseline=-.5ex]
\setbeamertemplate{itemize item}{\color{black}$\bullet$}
\setbeamertemplate{itemize subitem}{\color{black}$\bullet$}

\usetikzlibrary{shapes}
\usetikzlibrary{positioning}
\usetikzlibrary{automata}
\usepackage{amsmath,amssymb,amsfonts,amsthm}
\setbeamercovered{invisible} %% <--- I ADDED THIS
\newcommand{\red}{\textcolor{red}}
\newcommand{\blue}{\textcolor{blue}}
\newcommand{\purple}{\textcolor{purple}}
\newcommand{\brown}{\textcolor{brown}}
\newcommand{\cyan}{\textcolor{cyan}}
\newcommand{\real}{\ensuremath{\mathbb{R}}}
\newcommand{\y}{\ensuremath{\mathbf{y}}}
\newcommand{\black}{\color{black}}
\newcommand{\btheta}{\boldsymbol{\theta}}
\newcommand{\green}{\color{green}}
\newcommand{\word}[1]{\green{\textit{#1}\ }\black}
\newcommand{\lb}{\linebreak}
\newtheorem{com}{Comment}
\newtheorem{lem} {Lemma}
\newtheorem{prop}{Proposition}
\newtheorem{thm}{Theorem}
\newtheorem{defn}{Definition}
\newtheorem{cor}{Corollary}
\newtheorem{obs}{Observation}
\setcounter{tocdepth}{1}

\definecolor{UBCblue}{rgb}{0.04706, 0.13725, 0.26667}
\definecolor{UBCgray}{rgb}{0.3686, 0.5255, 0.6235}
\colorlet{verylightgray}{gray!10}
\setbeamercolor{palette primary}{bg=UBCblue,fg=white}
\setbeamercolor{palette secondary}{bg=darkgray,fg=white}
\setbeamercolor{palette tertiary}{bg=UBCblue,fg=white}
\setbeamercolor{palette quaternary}{bg=UBCblue,fg=white}
\setbeamercolor{structure}{fg=UBCblue} % itemize, enumerate, etc
\setbeamercolor{section in toc}{fg=UBCblue} % TOC sections
\setbeamercolor{subsection in head/foot}{bg=darkgray,fg=white}
\setbeamercolor{frametitle}{fg=UBCblue}
\setbeamercolor{title}{fg=UBCblue, bg=verylightgray}

\title[Class 10]{Introduction to Social Data Analytics \\
	\bigskip Class 10}
\author[Kaushik]{Arushi Kaushik}
\institute[UCSD]{arkaushi@ucsd.edu}
\date{23rd April 2019}

% update title block above

\begin{document}

\frame{\titlepage}

%\begin{frame}
%    \frametitle{Housekeeping}
%    \begin{itemize} \itemsep1em
%    \item Problem Set 2 due this morning
%    \item Quiz 3 due this Wednesday (9AM)
%    \item Stata plotting today, regression Wednesday
%    \item Review for midterm next week on Monday
%    \end{itemize}
%\end{frame}

\begin{frame}
\frametitle{Today: Plotting in Stata}
By the end of today's lecture, you should be able to:
\begin{itemize} \itemsep1em
	\item Create the following plots in Stata: scatter, line, bar, box, histogram
	\item Add elements to plots: titles, legends, fitted-lines, etc.
	\item Interpret elements of plots after creating them (e.g. quartiles in box plots)
	\item Demonstrate how to overlay multiple plots
\end{itemize} \bigskip
Open class10.do if you haven't already. 
\end{frame}

\begin{frame}
    \frametitle{Why do we present data visually?}
    \begin{itemize}
        \item The purpose of plotting is to demonstrate relationships between variables in your data. \pause
        \item Visual evidence is often more convincing than summary statistics by themselves.
    \end{itemize}
\end{frame}

\begin{frame}
    \frametitle{Why do we present data visually?}
    \begin{center}
    \includegraphics[scale=0.4]{chetty.png}
    \end{center}
\end{frame}

\begin{frame}
\frametitle{Plotting syntax}
The general syntax is always:
\begin{itemize}
        \item[] \texttt{graph twoway \textit{plottype varlist}, [\textit{options}]} 
\end{itemize} \bigskip
For example, you can create a scatter plot with:
\begin{itemize}
		\item[] \texttt{graph twoway scatter y x}
\end{itemize} \bigskip \pause
There are \textit{a lot} of different graphs to choose from in Stata. \\ \medskip
You don't always need \texttt{graph twoway} unless you are presenting different types of graphs on the same plot, but you \textit{always} need the graph type.
\end{frame}


\begin{frame}
\frametitle{Some plot types}
Today will we cover the following plot types (though there are several others):
\begin{itemize}
        \item line
        \item scatter
        \item histogram
        \item bar
        \item box
\end{itemize} 
\end{frame}

\begin{frame}
\frametitle{Line and Scatter Plots}
Syntax: \texttt{graph twoway line y1 [y2] [y3] ... x, [\textit{options}]}
\begin{itemize}
	\item Use \texttt{line} to show the change in a variable (y) over time (x)
	\item Use \texttt{scatter} to show the relationship between two variables where the points are not ordered (e.g. not ordered by time)
\end{itemize} \bigskip \pause
Some helpful options:
\begin{itemize}
	\item \texttt{lcolor(\textit{color})} 
	\item \texttt{legend(order(1 "Label 1" 2 "Label 2" ...)) }
\end{itemize} \bigskip \pause
Try it:
\begin{itemize}
	\item[] \texttt{sysuse uslifeexp, clear}
	\item[] \texttt{graph twoway line le\_male le\_female year, /// \\
		lcolor(red blue) legend(order(1 "Men" 2 "Women"))}
\end{itemize}
\end{frame}

\begin{frame}
\frametitle{Options that apply to all plots}
The order of options doesn't matter, but they all come after the comma:
\begin{itemize}
	\item \texttt{title("text")}
	\item \texttt{xtitle("text")}
	\item \texttt{ytitle("text")}
	\item \texttt{name(\textit{plotname}, replace)}
\end{itemize} \pause \bigskip
Try adding titles and names to your line plot:
\begin{itemize}
	\item[] \texttt{graph twoway line le\_male le\_female year\alert{,} /// \\
					lcolor(red blue) legend(order(1 "Men" 2 "Women")) /// \\
					\alert{title}("Life Expectancy for Men	vs Women") /// \\
					\alert{xtitle}("Year") /// \\
					\alert{ytitle}("Life Expectancy") /// \\
					\alert{name}(lifeexpmw, replace)}
\end{itemize}
\end{frame} 

\begin{frame}
\frametitle{Histograms}
Syntax: \texttt{histogram x}
\begin{itemize}
        \item Recall from Class 5: histograms display the distribution of the data
        \item Histograms require only one variable (the y-axis is the frequency)
\end{itemize} \bigskip \pause
Some helpful options:
\begin{itemize}
	\item \texttt{bin(\# of bins)} 
	\item \texttt{width(size)}
	\item \texttt{frequency}
	\item \texttt{kdensity}
\end{itemize} \bigskip \pause
Try it:
\begin{itemize}
	\item[] \texttt{sysuse citytemp, clear}
	\item[] \texttt{histogram tempjan, width(5) frequency kdensity}
\end{itemize}
\end{frame}

\begin{frame}
\frametitle{Bar Plots}
Syntax: \texttt{graph bar (function) x1 [x2]..., [over(\textit{varname})}] \\ \medskip
Example: \texttt{graph bar (mean) tempjan tempjul, over(region)} \bigskip
\begin{itemize}
	\item This creates a barplot depicting the mean temperatures in January and July by region
	\item \texttt{function} can be mean, median, sum, count, percent, etc.
\end{itemize} \bigskip \pause
Some helpful options:
\begin{itemize}
	\item \texttt{bar(1, color(\textit{color1})), bar(2, color(\textit{color2}))}
\end{itemize} \bigskip \pause
Try it:
\begin{itemize}
	\item[] \texttt{graph bar (mean) tempjan tempjul, over(region) /// \\
					legend(order(1 "January" 2 "July")) /// \\
					bar(1, color(green)) bar(2, color(purple)) }
\end{itemize}
\end{frame}

\begin{frame}
\frametitle{Box Plots}
Syntax: \texttt{graph box x1 [x2]..., [over(\textit{varname})}] \\ \medskip
\begin{itemize}
        \item Box plots describe statistical properties of data
        \item Usually used to compare across different groups \pause
        \item Definition: \textit{quartile} - what is it? \pause
        \item The line inside of the box is the median; the line at the top of the box is the third quartile, the line at the bottom of the box is the first quartile
        \item The \textit{interquartile range} (IQR) is: IQR =  Q3 - Q1 \pause
\end{itemize} \bigskip
Try it:
\begin{itemize}
	\item[] \texttt{graph box tempjan tempjul, over(region) /// \\
		legend(order(1 "January" 2 "July")) /// \\
		box(1, color(green)) box(2, color(purple)) }
\end{itemize}
\end{frame}


\begin{frame}
\frametitle{Finish up class10.do in your small groups.}
Here are the commands/operators we covered today:
\begin{columns}
	\begin{column}{0.5\textwidth}
		\begin{itemize}
			\item \texttt{graph twoway line}
			\item \texttt{graph twoway scatter}
			\item \texttt{histogram}
			\item \texttt{graph bar}
			\item \texttt{graph box}
		\end{itemize}
	\end{column}
	\begin{column}{0.5\textwidth}
		\begin{itemize}
			\item \texttt{title, xtitle, ytitle}
			\item \texttt{name}
			\item \texttt{legend(order(1 "label1" 2 "label2"))}
			\item \texttt{lcolor}
			\item \texttt{bin, width, frequency, kdensity}	
			\item \texttt{bar(1, color(color))}	
			\item \texttt{box(1, color(color))}	
		\end{itemize}
	\end{column}
\end{columns}
\end{frame}

\end{document}