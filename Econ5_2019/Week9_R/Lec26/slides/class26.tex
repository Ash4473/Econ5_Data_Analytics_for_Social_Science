%\documentclass[11pt]{beamer}
%\mode<presentation>
\documentclass[handout,compress]{beamer}
\usepackage{beamerthemedefault}
\usepackage{graphicx}
\usepackage{hyperref}
\usepackage{subfigure}
\usepackage{color}
\usepackage{multicol}
\usepackage{bm} 
\usepackage{tikz}
\usepackage{listliketab}

\usepackage{framed}
\definecolor{shadecolor}{RGB}{248,248,248}
\newenvironment{Shaded}{\begin{snugshade}}{\end{snugshade}}
\newcommand{\KeywordTok}[1]{\textcolor[rgb]{0.13,0.29,0.53}{\textbf{#1}}}
\newcommand{\DataTypeTok}[1]{\textcolor[rgb]{0.13,0.29,0.53}{#1}}
\newcommand{\DecValTok}[1]{\textcolor[rgb]{0.00,0.00,0.81}{#1}}
\newcommand{\BaseNTok}[1]{\textcolor[rgb]{0.00,0.00,0.81}{#1}}
\newcommand{\FloatTok}[1]{\textcolor[rgb]{0.00,0.00,0.81}{#1}}
\newcommand{\ConstantTok}[1]{\textcolor[rgb]{0.00,0.00,0.00}{#1}}
\newcommand{\CharTok}[1]{\textcolor[rgb]{0.31,0.60,0.02}{#1}}
\newcommand{\SpecialCharTok}[1]{\textcolor[rgb]{0.00,0.00,0.00}{#1}}
\newcommand{\StringTok}[1]{\textcolor[rgb]{0.31,0.60,0.02}{#1}}
\newcommand{\VerbatimStringTok}[1]{\textcolor[rgb]{0.31,0.60,0.02}{#1}}
\newcommand{\SpecialStringTok}[1]{\textcolor[rgb]{0.31,0.60,0.02}{#1}}
\newcommand{\ImportTok}[1]{#1}
\newcommand{\CommentTok}[1]{\textcolor[rgb]{0.56,0.35,0.01}{\textit{#1}}}
\newcommand{\DocumentationTok}[1]{\textcolor[rgb]{0.56,0.35,0.01}{\textbf{\textit{#1}}}}
\newcommand{\AnnotationTok}[1]{\textcolor[rgb]{0.56,0.35,0.01}{\textbf{\textit{#1}}}}
\newcommand{\CommentVarTok}[1]{\textcolor[rgb]{0.56,0.35,0.01}{\textbf{\textit{#1}}}}
\newcommand{\OtherTok}[1]{\textcolor[rgb]{0.56,0.35,0.01}{#1}}
\newcommand{\FunctionTok}[1]{\textcolor[rgb]{0.00,0.00,0.00}{#1}}
\newcommand{\VariableTok}[1]{\textcolor[rgb]{0.00,0.00,0.00}{#1}}
\newcommand{\ControlFlowTok}[1]{\textcolor[rgb]{0.13,0.29,0.53}{\textbf{#1}}}
\newcommand{\OperatorTok}[1]{\textcolor[rgb]{0.81,0.36,0.00}{\textbf{#1}}}
\newcommand{\BuiltInTok}[1]{#1}
\newcommand{\ExtensionTok}[1]{#1}
\newcommand{\PreprocessorTok}[1]{\textcolor[rgb]{0.56,0.35,0.01}{\textit{#1}}}
\newcommand{\AttributeTok}[1]{\textcolor[rgb]{0.77,0.63,0.00}{#1}}
\newcommand{\RegionMarkerTok}[1]{#1}
\newcommand{\InformationTok}[1]{\textcolor[rgb]{0.56,0.35,0.01}{\textbf{\textit{#1}}}}
\newcommand{\WarningTok}[1]{\textcolor[rgb]{0.56,0.35,0.01}{\textbf{\textit{#1}}}}
\newcommand{\AlertTok}[1]{\textcolor[rgb]{0.94,0.16,0.16}{#1}}
\newcommand{\ErrorTok}[1]{\textcolor[rgb]{0.64,0.00,0.00}{\textbf{#1}}}
\newcommand{\NormalTok}[1]{#1}

\usetheme{CambridgeUS}
\makeatletter
\makeatother
\usetikzlibrary{shapes,backgrounds}
\tikzstyle{cblue}=[circle, draw, thin,fill=cyan!20, scale=0.8]
\tikzstyle{qgre}=[rectangle, draw, thin,fill=green!20, scale=0.8]
\tikzstyle{rpath}=[ultra thick, red, opacity=0.4]
\tikzstyle{legend_isps}=[rectangle, rounded corners, thin,
                       fill=gray!20, text=blue, draw]
\usetikzlibrary{decorations.pathreplacing}
\tikzset{text/.default=}
%\tikzset{text/.align=0}
\tikzstyle{every picture}+=[remember picture]
\tikzstyle{na} = [baseline=-.5ex]
\setbeamertemplate{itemize item}{\color{black}$\bullet$}
\setbeamertemplate{itemize subitem}{\color{black}$\bullet$}

\usetikzlibrary{shapes}
\usetikzlibrary{positioning}
\usetikzlibrary{automata}
\usepackage{amsmath,amssymb,amsfonts,amsthm}
\setbeamercovered{invisible} %% <--- I ADDED THIS
\newcommand{\red}{\textcolor{red}}
\newcommand{\blue}{\textcolor{blue}}
\newcommand{\purple}{\textcolor{purple}}
\newcommand{\brown}{\textcolor{brown}}
\newcommand{\cyan}{\textcolor{cyan}}
\newcommand{\real}{\ensuremath{\mathbb{R}}}
\newcommand{\y}{\ensuremath{\mathbf{y}}}
\newcommand{\black}{\color{black}}
\newcommand{\btheta}{\boldsymbol{\theta}}
\newcommand{\green}{\color{green}}
\newcommand{\word}[1]{\green{\textit{#1}\ }\black}
\newcommand{\lb}{\linebreak}
\newcommand{\vitem}{\item}
\newtheorem{com}{Comment}
\newtheorem{lem} {Lemma}
\newtheorem{prop}{Proposition}
\newtheorem{thm}{Theorem}
\newtheorem{defn}{Definition}
\newtheorem{cor}{Corollary}
\newtheorem{obs}{Observation}
\setcounter{tocdepth}{1}

\definecolor{UBCblue}{rgb}{0.04706, 0.13725, 0.26667}
\definecolor{UBCgray}{rgb}{0.3686, 0.5255, 0.6235}
\colorlet{verylightgray}{gray!10}
\setbeamercolor{palette primary}{bg=UBCblue,fg=white}
\setbeamercolor{palette secondary}{bg=darkgray,fg=white}
\setbeamercolor{palette tertiary}{bg=UBCblue,fg=white}
\setbeamercolor{palette quaternary}{bg=UBCblue,fg=white}
\setbeamercolor{structure}{fg=UBCblue} % itemize, enumerate, etc
\setbeamercolor{section in toc}{fg=UBCblue} % TOC sections
\setbeamercolor{subsection in head/foot}{bg=darkgray,fg=white}
\setbeamercolor{frametitle}{fg=UBCblue}
\setbeamercolor{title}{fg=UBCblue, bg=verylightgray}

\title[Class 26]{Introduction to Social Data Analytics: \\
Class 26}
\author[Kaushik]{Arushi Kaushik}
\institute[UCSD]{arkaushi@ucsd.edu}
\date[]{}

\begin{document}

\frame{\titlepage}

\begin{frame}
\frametitle{Today: (Homemade) Functions in R}
By the end of today's lecture, you should be able to:
\begin{itemize}
	\item Describe how functions can save time and space while writing code
	\item Construct functions that perform basic calculations, e.g. the mean of a subset of data
	\item Identify the inputs and output in a function
\end{itemize} \bigskip
Please open class26.R and load AmericanTimeUse.RData if you haven't already.
\end{frame}

\begin{frame}[fragile]
 \frametitle{Share of housework in American households}
 \begin{center}
 \includegraphics[scale=.15]{images/housework.png}
 \end{center}
 \begin{itemize}
\item \alert{American Time Use Survey} (Bureau of Labor Statistics)
\item Measures the amount of \alert{time} people spend doing activities
\item What did you do in the \alert{last 24 hours}?
\item Coded into \alert{categories} by survey team
\end{itemize}
\end{frame}

\begin{frame}[fragile]
 \frametitle{American Time Use Survey}
\begin{verbatim}
# Read in Time use data
load("AmericanTimeUse.RData")
\end{verbatim}
Documentation for data is:
\url{https://www.bls.gov/tus/lexiconnoex0315.pdf}
\url{https://www.bls.gov/tus/atusintcodebk0315.pdf}
\end{frame}

\begin{frame}[fragile]
\frametitle{Your first function from PC26}
\begin{verbatim}
gender.mean <- function(gender) {
return(mean(time$household[time$TESEX == gender]))
}

gender.mean(1)
[1] 92.36724
gender.mean(2)
[1] 142.4052
\end{verbatim} \bigskip
How many inputs? How many outputs?

\end{frame}

\begin{frame}[fragile]
 \frametitle{Let's create our own `mean' function.}
One input: a vector of numbers \\
One output: the mean of the vector of numbers
\begin{verbatim}
my.mean <- function(vector){
  mu <- sum(vector)/length(vector)
  return(mu)
}

my.mean(time$household)
[1] 120.5359
mean(time$household)
[1] 120.5359
\end{verbatim}
\end{frame}



\begin{frame}[fragile]
 \frametitle{Add `variable' as an input:}
\begin{verbatim}
my.mean <- function(variable, data){
  mu <- mean(data[,variable])
  return(mu)
}

my.mean("household", time)
[1] 120.5359
my.mean("children", time)
[1] 33.27538
\end{verbatim}
\end{frame}


\begin{frame}[fragile]
 \frametitle{Add gender as an input}
 Say we wanted to calculate the mean for men/women separately. This is one way we could do it outside of a function:
\begin{verbatim}
table(time$TESEX) # 1 is male; 2 is female
    1     2 
74667 96175 
mean(time$household[time$TESEX == 1]) # men
[1] 92.36724
mean(time$household[time$TESEX == 2]) # women
[1] 142.4052
\end{verbatim} \pause
\alert{Your turn!} Write a function \texttt{gender.mean} that takes three inputs (variable, df, gender) and returns the mean of \texttt{df\$variable} for the specified gender.
\end{frame}

\begin{frame}[fragile]
 \frametitle{Add gender as an input}
\begin{verbatim}
gender.mean <- function(variable, data, gender){
   mu <- mean(data[data$TESEX == gender,variable])
   return(mu)
}

gender.mean("household", time, 1)
[1] 92.36724
gender.mean("household", time, 2)
[1] 142.4052
\end{verbatim}
\end{frame}

\begin{frame}[fragile]
 \frametitle{Add age to the Function}
 Basically we're going to do the same thing that we just did, but we are going to subset the data by \textit{both} gender and \alert{age}. \\
 \bigskip
 Work with your partner to write the function \texttt{gender.age.mean}. \\
 \bigskip
 Return the mean in \alert{hours} instead of minutes. 
\end{frame}


\begin{frame}[fragile]
 \frametitle{Add age to the Function}
\begin{verbatim}
gender.age.mean <- function(variable, data, gender, age){
   mu <- mean(data[data$TESEX == gender & 
              data$TEAGE == age, variable])/60 
   return(mu)
}

gender.age.mean("work", time, 1, 20)
[1] 3.340932
gender.age.mean("work", time, 2, 20)
[1] 2.611888
gender.age.mean("work", time, 1, 30)
[1] 4.480926
gender.age.mean("work", time, 2, 30)
[1] 2.845387
\end{verbatim}
\end{frame}

\begin{frame}[fragile]
 \frametitle{Putting it all together}
 Your job is to create a function \texttt{plot.by.age} that does the following:
 \begin{itemize}
	\item Takes inputs variable, dataframe
	\item Outputs a plot
	\begin{itemize}
		\item x-axis is years of age
		\item y-axis is mean hours spent on the specified activity (variable)
		\item Two data series, one for men, one for women
		\item Each point is the mean hours for a given age
	\end{itemize}
	\item Tip: you can use your created gender.age.mean() function inside your new function definition!
\end{itemize}
\end{frame}

\begin{frame}[fragile]
 \frametitle{You final plot should look like this}
 \begin{center}
\includegraphics[scale=.35]{images/PlotFinal.pdf}
\end{center}
\end{frame}


\end{document}











